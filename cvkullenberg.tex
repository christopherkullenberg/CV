%%%%%%%%%%%%%%%%% PREAMBLE %%%%%%%%%%%%%%%%%%%%%%%%%%%%
%Change the font size of your document - 10pt, 12.1pt, etc. letterpaper in original
\documentclass[a4paper,11pt,oneside]{article}
%\setcounter{section}{-1}  %Use this if you want the first page to be in Table of Contents
\usepackage[utf8]{inputenc}
\usepackage{setspace}
\usepackage{hyperref} % Use this with normal dvipdf
\usepackage{graphicx}
\usepackage{pdfpages}
\usepackage{tikz}
\graphicspath{ {/}} %upload your signature to this file
\usepackage[left=1in, right=1in, bottom=1.25in, top=1.25in]{geometry}
\renewcommand{\contentsname}{Contents}
%%%%%%%%%%%%%%%%% END OF PREAMBLE %%%%%%%%%%%%%%%%%%%%%

\begin{document}

%%%%%%%%%%%%%%%%%%%%%%%%%%%%%%%%%%%%%%%%%%%%%%%%%%%%%%%
%%%%%%%%%%%%%%%%% NAME OF APPLICANT %%%%%%%%%%%%%%%%%%%
%%%%%%%%%%%%%%%%%%%%%%%%%%%%%%%%%%%%%%%%%%%%%%%%%%%%%%%
% Use these two linesif First page should be in Table of Contents
  %\renewcommand{\thesection}{}
  %\noindent  \LARGE{\textbf{\section[Summary CV]{Christopher Kullenberg, CV}}}  \\
% also uncomment \setcounter{section}{-1} in the preamble.

\noindent  \LARGE{\textbf{Christopher Kullenberg, CV}}  \\
\vspace{-2ex}
\hline
\normalsize

%%%%%%%%%%%%%%%%%%%%%%%%%%%%%%%%%%%%%%%%%%%%%%%%%%%%%%%
%%%%%%%%%%%%%%%%% CONTACT INFORMATION %%%%%%%%%%%%%%%%%
%%%%%%%%%%%%%%%%%%%%%%%%%%%%%%%%%%%%%%%%%%%%%%%%%%%%%%%
\begin{center}
\begin{tabular}{l l}
  %original value of hspace is 1in (for letter paper), changed to .5 for a4paper
 University of Gothenburg    & \hspace{.2in} \href{mailto:christopher.kullenberg@gu.se}{christopher.kullenberg@gu.se} \\
 Department of Philosophy, Linguistics \& Theory of Science    & \hspace{.2in}  \href{http://christopherkullenberg.se}{christopherkullenberg.se}   \\
 Box 200             & \hspace{.2in}  \href{https://github.com/christopherkullenberg}{github.com/christopherkullenberg}   \\
 405 30 Göteborg & \hspace{.2in} Phone: +46 (0) 735-083022 \\
 SWEDEN & \hspace{.2in} ORCID: \href{http://orcid.org/0000-0002-1577-3570}{0000-0002-1577-3570} \\
 \\ (Hyperlinks available if your PDF viewer supports it) & \hspace{.2in}  \textbf{Updated: \today} \\
\end{tabular}
\end{center}

\vspace{1em}

%%%%%%%%%%%%%%%%%%%%%%%%%%%%%%%%%%%%%%%%%%%%%%%%%%%%%%%
%%%%%%%%%%%%%%%%% MAIN BODY %%%%%%%%%%%%%%%%%%%%%%%%%%%
%%%%%%%%%%%%%%%%%%%%%%%%%%%%%%%%%%%%%%%%%%%%%%%%%%%%%%%

\noindent \begin{tabular}{@{} l l}
 \Large{Education}    & \textbf{University of Gothenburg} \\
     & - \href{http://files.christopherkullenberg.se/doctoraldegreeonline.pdf}{Ph.D., Theory of Science, 2012.} \\
     & - \href{http://files.christopherkullenberg.se/examengrundutbildningonline.pdf}{M.A., Theory of Science, 2005}. \\
     & - \href{http://files.christopherkullenberg.se/examengrundutbildningonline.pdf}{B.A., Media \& Communication Science, 2004}. \\
     & \\

 \Large{Dissertation}    & \href{http://hdl.handle.net/2077/28807}{--- (2012)\emph{The Quantification of Society - A Study of a Swedish Research Institute and}} \\
    & \parbox{5.0in}{\href{http://hdl.handle.net/2077/28807}{\emph{Survey-based Social Science}.}}
    & \\
 \Large{Research}    & \textbf{University of Gothenburg} \\
     & - Researcher 50\%, 20140601-20180601. \\
     & Project: ``Taking Science to the Crowd: Researchers, Programmers and Volunteer  \\
     & Contributors Transforming Science Online.''\\
     & Marianne \& Marcus Wallenberg Foundation\\
     & PI: \href{mailto:dick.kasperowski@gu.se}{Dick Kasperowski}, Grant number: \href{https://www.wallenberg.com/MMW/projektanslag-2013}{MMW 2013.0020}.\\
     & - Researcher 50\%, 20140101-20161231. \\
     & Project: ``The Co-production of Social Science and Society:   \\
     & The Case of Happiness studies.''\\
     & Swedish Research Council (Vetenskapsrådet)\\
     & PI: \href{mailto:margareta.hallberg@gu.se}{Margareta Hallberg}, Grant number: \href{http://vrproj.vr.se/detail.asp?arendeid=90421}{2012-1117}.\\
     & - Researcher 50\%, October 2012 – March 2013.\\
     & Project: ``Subcultures on the Net: Resistance and Engagement in Knowledge, \\
     &  Practices'', LETStudio, Pilot project.
     & \\
  \Large{Teaching}   & \textbf{University of Gothenburg} \\
     & - \href{http://files.christopherkullenberg.se/anstallningarGU.pdf}{Instructor, part time, 2014-2015}.\\
     & - \href{http://files.christopherkullenberg.se/studierektoronline.pdf}{Study Administrator, spring semester 2014}. \\
     & - \href{http://files.christopherkullenberg.se/anstallningarGU.pdf}{Adjunct Lecturer, part time, 2006-2013}. \\
     &\textbf{University West} \\
     & - \href{http://files.christopherkullenberg.se/universitywest.pdf}{Instructor, part time, 2004-2006}. \\
     &\textbf{University of Borås} \\
     & - \href{http://files.christopherkullenberg.se/anstallningsbeslutboras.pdf}{Lecturer, part time, 201708-201802}. \\
     & \\
 \Large{Awards}   & \textbf{Swede of the Year}, Awarded by \href{http://www.fokus.se/2011/12/med-datorn-som-vapen/}{Fokus Magazine}, 2011.
                  &\\
                  & \textbf{Best Open Science}, \href{http://www.mynewsdesk.com/se/pressreleases/1st-annual-open-knowledge-awards-announces-the-winners-1684562}{Open Knowledge Awards, 2016}.\\
                  & \\
  \Large{Languages}   & Swedish (native), English (advanced), German (intermediate), \\
\Large{and Skills}    & Python programming language (\href{https://courses.edx.org/certificates/f9c30b3913be4004b95813db59432509}{MITx 6.00.1x Introduction to Computer Science}).  \\
\end{tabular}
%end


\newpage
%%%%% TABLE OF CONTENTS %%%%
\tableofcontents
\newpage




%%%%%%%%%%%%%%%%%%%%%%%%%%%%%%%%%%%%%%%%%%%%%%%%%%%%%%%
%__________                                          .__
%\______   \ ____   ______ ____ _____ _______   ____ |  |__
% |       _// __ \ /  ___// __ \\__  \\_  __ \_/ ___\|  |  \
% |    |   \  ___/ \___ \\  ___/ / __ \|  | \/\  \___|   Y  \
% |____|_  /\___  >____  >\___  >____  /__|    \___  >___|  /
%        \/     \/     \/     \/     \/            \/     \/
%
%%%%%%%%%%%%%%%%%%%%%%%%%%%%%%%%%%%%%%%%%%%%%%%%%%%%%%%
%%%%%%%%%% RESEARCH SUMMARY %%%%%%%%%%%%%%%%%%%%%%%%%%%
%%%%%%%%%%%%%%%%%%%%%%%%%%%%%%%%%%%%%%%%%%%%%%%%%%%%%%%
% TODO: Fundera på om jag ska ta med Dubrovnik-stipendiet. Ligger i ``files''


\renewcommand{\thesection}{\arabic{section}}
\clearpage
\newpage
\setlength\parindent{0cm}
\section{Research portfolio}
%\noindent  \LARGE{\textbf{Research}}  \\

    \subsection{Overview}
    The scope of my research can be divided into four major subject areas that both
    deepen and widen my disciplinary integration with Theory of Science and Science \&
    Technology Studies (STS). The two first areas are broad inquiries into the social,
    historical and epistemic conditions that shape the phenomena of citizen science and quantitative
    social sciences, respectively. This research has been pursued within long-term,
    externally funded research projects, and have made use of the theoretical resources
    that are usually regarded as part of the field of ``Science studies''. Moreover, the main
    output of these projects has been scholarly, although the research on citizen
    science has attracted additional interest also from a broader community of volunteer
    contributors to science and the mass media.\\

    The third area, concerning the social impact of digital technologies, originated as a
    side-project where I begun analyzing the rapid changes in the media landscape
    by applying a general philosophical approach to technology. This interest has
    had its major output in the shape of populariziations and collaborations with
    a broader public, and I have frequently written for mass media outlets on these
    issues as well as given numerous interviews.\\

    Finally, in recent years, I have worked to develop digital methods to be able to
    take advantage of the rich data sources, which in recent times have become
    easilys accessible to researchers, sometimes under the umbrella term ``digital humanities''.
    Here I have collaborated with programmers and people working in the library and
    archive sectors. Besides writing on the theoretical implications
    of digital methods, I have also published software and repositories of digital
    material available to the public and other researchers.\\

    As these four areas are quite diverse, I will elaborate on them separately
    below, and list key publications for each field. \\


    \subsubsection{Citizen Science \& Scientific Citizenship}
    As a researcher in the project ``Taking Science to the Crowd: Researchers,
    Programmers and Volunteer Contributors Transforming Science Online'' I have had
    the benefit of studing the emerging trend of citizen science as a trans-disciplinary
    team effort, working together with researchers from applied IT, pedagogics and engineering. \\

    With my colleague Dick Kasperowski, I have published the largest meta-study
    to date of the phenomenon of citizen science (1), a study that was well-received also by citizen
    scientists themselves, and was reported in the mass media. We used
    scientometric methods to discover multiple strands of research in
    academic literature to show how two forms of citizen science had emerged
    independently in the natural and social sciences.\\

    Moreover, I have worked on the theoretical problems of representation and reference
    in activist-oriented citizen science. In the first issue of the \emph{Journal of Resistance Studies}}
     I published a study (2) that analysed how grassroot citizen scientists use scientific
    methods and knowledge to address local environmental problems. I argued that this
    practice amalgamates scientific methodologies with political struggles of
    representing local communities, leading to a sort of ``strategic universalism'',
    in which science becomes an important resource for action. \\

    Furthermore, in collaboration with \emph{Public \& Science} (Vetenskap och allmänhet) and the
    European Commission's ``Researcher's Night'' I have planned and will execute the
    citizen humanities project \emph{\href{http://v-a.se/2016/05/efterlyses-skolelever-till-banbrytande-forskningsprojekt/}{Forskarfredag: Anslagstavlan}}.
    It involves more that one thousand school children collecting bulletin board
    messages across Sweden using mobile phone cameras. The data will be analysed
    during the autumn 2016 by a transdisciplinary research team at the universities
    of Gothenburg and Stockholm.\\

    As citizen science has gained a lot of attention recently, I have also participated in
    outreach activities, for example by participating in the monthly chat
    \emph{\href{http://blogs.plos.org/citizensci/2016/05/17/coops-scoop-citizen-science-practitioners-walk-the-walk-with-open-science-on-the-next-citscichat/}{#CitSciChat}},
    commented Citizen Science on the TV show \emph{\href{https://youtu.be/vG8sZQnU7mU?t=18m1s}{``Vetenskapens värld.''}} and
    given several interviews. \\


      \noindent  \emph{Main publications} \\
      (1) ---, \& Dick Kasperowski (2016) ``What Is Citizen Science? – A Scientometric Meta-Analysis'', \emph{PLoS ONE}, 11(1): e0147152. DOI: \href{http://dx.doi.org/10.1371/journal.pone.0147152}{10.1371/journal.pone.0147152}.\\
      (2) --- (2015) ``Citizen Science as Resistance: Crossing the Boundary Between Reference and Representation'', \href{https://gup.ub.gu.se/publication/218601-citizen-science-as-resistance-crossing-the-boundary-between-reference-and-representation}{\emph{Journal of Resistance Studies}}, 1(1).\\

    \subsubsection{The Co-production of Social Science and Society}
    Already as an undergraduate student I have worked on the problem of how the
    social sciences are co-constructed with social and historical conditions. As a
    Ph.D-candidate I approached this problem in my dissertation (2) by conducting a
    case study of a large Swedish social scientific survey, the SOM-institute. Here I
    combined a historical perspective that analysed the role of social science in the
    modern welfare state (1) with a close-up view of the practice of quantification and
    survey field work. I argued that the practice of quantifying society was closely
    intertwined with a wider societal demand for knowledge in numbers, and that the social
    sciences worked as one of several centers of calculation in the modern welfare state.\\

    After defending my Ph.D-thesis, I continued this line of research in the project
    ``The Co-production of Social Science and Society: The Case of Happiness studies.''.
    Here I approached a specific field of research in the social and medical sciences:
    ``Happiness studies''. With my colleague Gustaf Nelhans, I published a scientometric analysis
    of the historical emergence of happiness studies, using cited references in journal
    publications as a method for mapping the disciplinary origins and topical foci of
    what is today known as happiness research. \\

    % ahmed abf
    % BNP och lycka-artikeln

      \noindent  \emph{Main publications} \\
      ---, \& Nelhans Gustaf (2015) ``The happiness turn? Mapping the emergence of happiness studies using cited references'', \emph{Scientometrics}, Volume 103, Issue 2, Page 615-630, DOI: \href{http://dx.doi.org/10.1007/s11192-015-1536-3}{10.1007/s11192-015-1536-3}.\\
      --- (2012) \emph{The Quantification of Society. A Study of a Swedish Research Institute
      and Survey-Based Social Science}, Department of Philosophy, Linguistics and Theory of Science, University of Gothenburg, Doctoral dissertation,  \href{https://gupea.ub.gu.se/handle/2077/28807}{ISBN 978-91-628-8458-1}. 213 pages.\\
      --- (2011) ``Sociology in the Making: Statistics as a Mediator between the Social sciences, Practice and the State'', in Ann Rudinow Saetnan, Heidi Mork Lomell, Svein Hammer, \emph{The Mutual Construction of Statistics and Society}, Routledge.\\


    \subsubsection{The Social Impact of Digital Technologies}
    My research on the social impact of digital technologies appeared at first as a theoretical
    interest. With Karl Palmås, I published the article ``Contagiontology'' in 2009 (3),
    where we analysed the relationship between new forms of surveillance and emerging
    marketing practices. We showed how digital technologies enable a new social diagram
    of control, where meta-data had begun to play a major role both in mass-surveillance
    and in viral marketing. I expanded this perspective to include also how warfare
    was transformed in the digital realm (2) and was frequently invited as a commentator
    on surveillance, both at conferences and by the mass media.\\

    In the project \href{http://letstudio.gu.se/studio-2/subcultures-on-the-net}{``Subcultures on the Net: Resistance and engagement in knowledge practices''},
    a pilot study funded by the \href{http://letstudio.gu.se/}{LETStudio}, I
    approached internet activism and the strategies they used to promote a free and
    open internet (1), returning to a theme that I had analysed a few years earlier (4).\\

    This are of research has been performed in close proximity to communities online.
    In 2011, as a participant in the activist group Telecomix, I was
    awarded the prize \href{http://www.fokus.se/2011/12/med-datorn-som-vapen/}{``Swede of the year''} by Fokus Magazine
    for making public the role of the internet in the Arab Spring. \\

    Since 2008, I have frequently commented on issues of digital technologies
    in the mass media in more than 50 interviews (including international outlets), and have been invited as a speaker by
    the Green Group in the European Parliament, Deutsche Welle, The Council of Europe,
    the Swedish Embassy in Cairo, Reporters Sans Frontieres in Geneva, and by multiple conferences.
    I have written more than 25 opinion pieces in newspapers on digital technologies and society.\\

      \noindent  \emph{Main publications} \\
      (1)--- (2013) ``Resistance in hybrid networks: The case of Telecomix'', in Bueti, Federica (ed.), \emph{Move… ment}, London: Book Works.\\
      (2)--- (2009) ``The Social Impact of IT – Surveillance and Resistance in Present Day Conficts'', \href{http://www.fiff.de/publikationen/fiff-kommunikation/fk-2009/fiff-ko-1-2009/fiko_1_2009_kullenberg.pdf}{\emph{FiFF-Kommunikation}}, 09(1).\\
      (3)---, \& Palmås, Karl (2009) ``Contagiontology'', \href{http://www.eurozine.com/articles/2009-03-09-kullenberg-en.html}{\emph{Eurozine}}. First published in \emph{Glänta} 2008(4) as ``Smittontologi''.\\
      (4)--- (2009) ``Den panspektriska tidsålderns motståndsstrategier'', i Lilja, Mona \& Vinthagen Stellan (red) \emph{Motstånd}, Malmö: Liber AB.\\


    \subsubsection{Digital Methods in the Social Sciences and Humanities}
    As a consequence of my interest in the social impact of digital technologies,
    I began to develop a number of methodological inquiries. My approach was
    at first theoretical, and I began to introduce sociological concepts from the
    19th century sociologist Gabriel de Tarde (3, 4) to find a new conceptual
    ground for analyzing the vast digital material, sometimes referred to as the
    ``data deluge''.\\

    However, these theoretical discussions needed to be combined with a methodological and
    empirical approach. While digital material, or ``found data'', is widely and cheaply
    available, both in open sources and in archives and databases, it also requires
    technical skills to be turned into systematic scientific data. I approached the
    field of scientometrics and used this methodology in two large studies (5, 6).\\

    To make full of digital methods, however, it is necessary to be able to write
    your own programs and to know how to build scientifically valid databases. In
    2015, and still developing, I learned the Python programming language and began
    writing software and scripts for my ongoing research (see for example
    \href{https://github.com/christopherkullenberg}{github.com/christopherkullenberg}).
    I also learned how to work with SQL databases and Lucene-based search engines for big data.
    Some of these projects I also equipped with user interfaces, such as
    \href{https://genuskollen.se/}{genuskollen.se}, and I am currently working
    on a prototype for rendering all government whitepapers (Statens Offentliga Utredningar)
    searchable in an open database (see demo with limited dataset at
     \href{http://offentligautredningar.flov.gu.se/}{offentligautredningar.flov.gu.se}).\\

     Lastly, in digital methods, there is a pressing concern for a re-definition
     of research ethics. As ``found data'' is often rich in personal information
     and can be easily abused when not handled correctly, I have addressed these
     issues with my colleagues (7) to advance what we call the ``data minimization principle'',
     meaning that the researcher should craft his or her data collection tools
     to collect and process only the necessary material used for the purpose of
     the study at hand.\\

     In 2017 I will publish a large Swedish-language edited volume on digital methods in the social
     sciences and humanities (2), which includes chapters written by prominent Swedish
     scholars from several disciplines. The target audience is both students,
     researchers and journalists.\\

      \noindent  \emph{Main publications} \\
      (1)--- (2016) ``Sökandet efter den digitala politiken'', \href{https://humanit.hb.se/article/view/504/572}{\emph{Human IT}}, 13(2): 34-46.\\
      (2)--- (2017, forthcoming, eds.) ``Digitala metoder för samhällsvetenskap och humaniora''.\\
      (3)--- (2015) ``Internetövervakning som metod'', \href{http://gup.ub.gu.se/records/fulltext/220443/220443.pdf}{\emph{Ikaros}}, 2015(1), pp 9–11.\\
      (4)--- (2012) ``Introduktion'', in \emph{Tre Klassiska Texter – Émile Durkheim, Gabriel Tarde, Max Weber}, fackgranskad av Christopher Kullenberg, Göteborg: Korpen Koloni.\\
      (5)---, \& Nelhans Gustaf (2015) ``The happiness turn? Mapping the emergence of happiness studies using cited references'', \emph{Scientometrics}, Volume 103, Issue 2, Page 615-630, DOI: \href{http://dx.doi.org/10.1007/s11192-015-1536-3}{10.1007/s11192-015-1536-3}.\\
      (6) ---, \& Dick Kasperowski (2016) ``What Is Citizen Science? – A Scientometric Meta-Analysis'', \emph{PLoS ONE}, 11(1): e0147152. DOI: \href{http://dx.doi.org/10.1371/journal.pone.0147152}{10.1371/journal.pone.0147152}.\\
      (7) (in press) Hård af Segerstad, Y. ; Howes, C. ; Kasperowski, D., Kullenberg, C. (2017) ``Studying closed communities on-line: digital methods and ethical considerations beyond informed consent and pseudonymity''. Zimmer, M. & Kinder-Kurlanda, K. (Eds.). \emph{Internet Research Ethics for the Social Age: New Cases and Challenges}. Peter Lang: Digital Formations (Steve Jones, series editor). (Peer-reviewed book chapter, accepted by publisher) \\

%%%%%%%%%%%%%%%%%%%%%%%%%%%%%%%%%%%%%%%%%%%%%%%%%%%%%%%
%%%%%%%%%%% FUTURE RESEARCH %%%%%%%%%%%%%%%%%%%%%%%%%%%
%%%%%%%%%%%%%%%%%%%%%%%%%%%%%%%%%%%%%%%%%%%%%%%%%%%%%%%


%\subsection{Future research}
    % Digital methods
    % Citizen science and Data citizenship, open science osv.



%%%%%%%%%%%%%%%%%%%%%%%%%%%%%%%%%%%%%%%%%%%%%%%%%%%%%%%
%%%%%%%%%%%%%%% PUBLICATIONS %%%%%%%%%%%%%%%%%%%%%%%%%%
%%%%%%%%%%%%%%%%%%%%%%%%%%%%%%%%%%%%%%%%%%%%%%%%%%%%%%%
%   1. Skrifter Skrifter förtecknas endast vid en punkt. Ange utgivningsår, ISBN/ISSN samt antal-et sidor.
%   Den egna insatsen ska tydligt särskiljas vid sampublicering.
%   a)Monografier
%   b)Antologier-refereegranskat bidrag: Avser artikel eller konferenspublikation som granskats av oberoende referenter (peer review) -redaktörsgranskat bidrag-redaktörskap -övrigt
%   c)Artiklar
%    -refereegranskad: Avser artikel som granskats av oberoende re-ferenter (peer review).
%    -övrig vetenskaplig: Avser att artikelns innehåll är av vetenskaplig karaktär och riktar sig mot vetenskapssamhället, men att artikeln ej peer review granskats.
%   d)Opublicerade men av utgivare accepterade manus
%   e)Övrigt-Används endast när ingen annan publikationstyp är tillämplig. Hit hör t.ex. översättningar, källpublikationer, utredningar, rapporter m.m. Avser även konferensbidrag som har granskats av referenter men inte publicerats i officiell proceeding eller i konferensmed-delande

\clearpage
\setlength\parindent{0cm}
\subsection{Publications}
%\noindent  \LARGE{\textbf{Publications}}  \\

\subsubsection{Monographs}
    --- (2012) \emph{The Quantification of Society. A Study of a Swedish Research Institute
    and Survey-Based Social Science}, Department of Philosophy, Linguistics and Theory of Science, University of Gothenburg, Doctoral dissertation,  \href{https://gupea.ub.gu.se/handle/2077/28807}{ISBN 978-91-628-8458-1}. 213 pages.\\
    --- (2010) \emph{Det Nätpolitiska Manifestet}, Stockholm: Ink bokförlag, ISBN 9789197846912. 80 pages.

\subsubsection{Book chapters}
  \noindent  \emph{Peer reviewed} \\
  --- Hård af Segerstad, Y. ; Howes, C. ; Kasperowski, D. \& Kullenberg, C. (2017, in press) ``Studying closed communities on-line: digital methods and ethical considerations beyond informed consent and pseudonymity''. Zimmer, M. & Kinder-Kurlanda, K. (Eds.). \emph{Internet Research Ethics for the Social Age: New Cases and Challenges}. Peter Lang: Digital Formations (Steve Jones, series editor).. : Peter Lang. \\
  --- (2011) ``Sociology in the Making: Statistics as a Mediator between the Social sciences, Practice and the State'', in Ann Rudinow Saetnan, Heidi Mork Lomell, Svein Hammer, \emph{The Mutual Construction of Statistics and Society}, New York: Routledge.\\

  \noindent \emph{Editorial review} \\
  %--- Hård af Segerstad, Y. ; Kasperowski, D. ; Kullenberg, C. (2016) ``Att studera slutna grupper på nätet: Digitala metoder och etiska överväganden''. in Kullenberg, Christopher (Eds.) \emph{Digitala metoder för samhällsvetenskap och humaniora}. \\
  --- (2014) ``Verkon tunnelit ja niiden kaivajat'', in Brunila Mikael \& Kallio Kimmo (eds.), \emph{Verkko suljettu – Internet ja avoimuuden rajat}, Helsinki: Into kustannus, ISBN 978-952-264-248-6.\\
  --- (2013) ``Resistance in hybrid networks: The case of Telecomix'', in Bueti, Federica (ed.), \emph{Move… ment}, London: Book Works.\\
  --- (2009) ``Den panspektriska tidsålderns motståndsstrategier'', in Lilja, Mona \& Vinthagen Stellan (eds) \emph{Motstånd}, Malmö: Liber AB.\\
  --- (2008) ``Tre frågor om sociologins objektivitet och förhållande till naturvetenskaperna'', in Hallberg, Margareta (eds), \emph{Vi vet något – Festskrift till Jan Bärmark}, Göteborg: Institutionen för idéhistoria och vetenskapsteori, ISBN 978–91–976239–1–9.\\

\subsubsection{Journal articles}
  \noindent  \emph{Peer reviewed} \\
  % Lägg till Marisas artikel när formell accept
  % Fixa länk till valuation studies.
  ---, \& Nelhans, Gustaf (2017, in press) ``Measuring Welfare beyond GDP --- Objective and Subjective Indicators in Sweden, 1968-2015'', \emph{Valuation Studies}, 5(1).\\
  ---, \& Dick Kasperowski (2016) ``What Is Citizen Science? – A Scientometric Meta-Analysis'', \emph{PLoS ONE}, 11(1): e0147152. DOI: \href{http://dx.doi.org/10.1371/journal.pone.0147152}{10.1371/journal.pone.0147152}.\\
  --- (2015) ``Citizen Science as Resistance: Crossing the Boundary Between Reference and Representation'', \href{https://gup.ub.gu.se/publication/218601-citizen-science-as-resistance-crossing-the-boundary-between-reference-and-representation}{\emph{Journal of Resistance Studies}}, 1(1).\\
  ---, \& Nelhans, Gustaf (2015) ``The Happiness Turn? Mapping the Emergence of Happiness Studies using Cited References'', \emph{Scientometrics}, Volume 103, Issue 2, Page 615-630, DOI: \href{http://dx.doi.org/10.1007/s11192-015-1536-3}{10.1007/s11192-015-1536-3}.\\

  \noindent  \emph{Non-peer review} \\
    --- (2016) ``Sökandet efter den digitala politiken'', \href{https://humanit.hb.se/article/view/504/572}{\emph{Human IT}}, 13(2): 34-46 (book review).\\
    --- (2015) ``Internetövervakning som metod'', \href{http://gup.ub.gu.se/records/fulltext/220443/220443.pdf}{\emph{Ikaros}}, 2015(1), pp 9–11.\\
    --- (2011) ``Bortom Wikileaks - Om nätläckans infrastruktur'', \emph{Ord \& Bild}, 2011(1).\\
    --- (2009) ``The Social Impact of IT – Surveillance and Resistance in Present Day Conficts'', \href{http://www.fiff.de/publikationen/fiff-kommunikation/fk-2009/fiff-ko-1-2009/fiko_1_2009_kullenberg.pdf}{\emph{FiFF-Kommunikation}}, 09(1).\\
    ---, \& Palmås, Karl (2009) ``Contagiontology'', \href{http://www.eurozine.com/articles/2009-03-09-kullenberg-en.html}{\emph{Eurozine}}. First published in \emph{Glänta} 2008(4) as ``Smittontologi''.\\
    --- (2008) ``Robotar och krig'', \emph{Ord \& Bild}, 2008(5).\\

   \noindent   \emph{Pre-prints}\\
     --- Ponti, M., T Hillman, D Kasperowski, C Kullenberg (2017) ``Getting it Right or Being Top Rank: Games in Citizen Science'', Socarxiv,
      \href{https://osf.io/preprints/socarxiv/3qrnc/}{https://osf.io/preprints/socarxiv/3qrnc/} \\
     --- Kasperowski, D., Kullenberg, C., Mäkitalo, Å. (2017) ``Embedding Citizen Science in Research: Forms of engagement, scientific output and values for science, policy and society'', Socarxiv,  \href{https://osf.io/preprints/socarxiv/tfsgh/}{https://osf.io/preprints/socarxiv/tfsgh/}\\


\subsubsection{Other publications}

   \noindent \emph{Reports}\\
   --- Björkvall, Anders, Johan Järlehed, Christopher Kullenberg, Helle Lykke Nielsen,
Andreas Nord, Tove Rosendal, Sara Van Meerbergen & Gustav Westberg (2017) \href{https://www.forskarfredag.se/filer/ff2016-anslagstavlan-slutrapport.pdf}{Slutrapport
   Anslagstavlan - Forskarfredags Massexperiment 2016, VA-rapport 2017:1}, Red. Fredrik Bronéus,
   Stockholm: Vetenskap och Allmänhet, ISSN 1653-6843.\\
   --- Kullenberg, Christopher (2017) \href{http://www.slu.se/globalassets/ew/subw/lifewatch/publikationer/slw-summary-report-web-170622.pdf}{Open data - buzz word or virtual opportunities?},
   in \emph{Swedish LifeWatch – a national e-infrastructure for biodiversity data}, ArtDatabanken SLU, ISBN
978-91-87853-17-3.\\



  \noindent \emph{Preface and review of translation}\\
    --- (2012) ``Introduktion'', in \emph{Tre Klassiska Texter – Émile Durkheim, Gabriel Tarde, Max Weber}, fackgranskad av Christopher Kullenberg, Göteborg: Korpen Koloni.\\

  \noindent \emph{Conference contributions, peer reviewed}\\
  – Kullenberg, C. (2017) ``Talking about #citizenscience'', \href{http://socav.gu.se/english/research/third-nordic-science-and-technology-studies-conference}{Third Nordic Science and Technology Conference, Göteborg, May 31-June2.}\\
  - Kasperowski, D. ; Kullenberg, C. (2016) ``Citizen Humanities: Configuring Interpretation and Perception for Participation''. \emph{Citizen Science – Innovation in Open Science, Society and Policy}, 19–21 May 2016, Berlin, 26-27.\\
  - Nelhans, G. ; Kullenberg, C. (2016) ``Happiness as a Valuation of Nations: From Margin to Indicator''. \emph{4S/EASST Conference}, Barcelona 2016.\\
  - Ponti, M; Hagen, N; Hillman, T; Kasperowski, D; Kullenberg, C; Stankovic, I (2015) ``Designing Futures for Learning in the Crowd: New Challenges and Opportunities for CSCL''. In: Lindwall, O., Häkkinen, P., Koschman, T. Tchounikine, P. & Ludvigsen, S. (Eds.) (2015). \emph{Exploring the Material Conditions of Learning: The Computer Supported Collaborative Learning (CSCL) Conference}, Vol. 2 s. 885-888\\
  - Kasperowski, D; Hagen, N; Kullenberg, C; Walford, A; Smith, K, V; Liborion, M; Prutzer N & Hamid, S, T (2015) ``Joining Reference and Representation — Citizen Science as Resistance Practice''. \emph{Society for Social Studies of Science 2015 Annual Meeting}, November 11-14 Denver, Colorado.\\
  --- (2008) ``The Panspectrocist Abstract Machine - Machinic Phyla, Territorialities and Social Diagrams''. \emph{The First International Deleuze Studies Conference}, 1 Monday: 11/08/08.\\
  --- (2008) ``The Objects and Objections of the Social Sciences – The Re-discovery of Epistemic Practice as a Tool for Enhancing Public Participation''. \emph{Ironists, Reformers or Rebels? The Role of the Social Sciences in Participatory Policy Making}, June 26-27th 2008 at Collegium Helveticum, UZH/ETH Zürich, Switzerland.\\
  --- (2008) ``From State Sociology to Centres of Calculation: the Swedish case''. \emph{Acting with Science, Technology and Medicine 4S-EASST 2008 Rotterdam}, the Netherlands, August 20-23 2008. . 367-368.\\

  \noindent  \emph{Research notes} \\
    --- (2017) ``Bibliometric probing of the concept 'open science' - a notebook'', \\
    \href{https://github.com/christopherkullenberg/openscienceliterature/}{https://github.com/christopherkullenberg/openscienceliterature/}. \\

 \noindent  \emph{Unpublished manuscripts}  \\
   ---(eds) (2017) ``Digitala metoder för samhällsvetenskap och humaniora'', (Edited volume). \\



  \noindent \emph{Popularizations in news media} \\
    --- (2015) ``Tankesprång som kopplar ihop världen'', \href{http://www.svd.se/tankesprang-som-kopplar-ihop-varlden}{\emph{SvD Under Strecket}}, 4 September, 2015.\\
    --- (2014) ``Gazakonfliktens underjordiska liv'', \href{http://www.svd.se/kultur/understrecket/gazakonfliktens-underjordiska-liv_3776810.svd}{\emph{SvD Under Strecket}}, 26 Juli, 2014.\\
    --- (2011) ``Internet nära oss'', \href{http://www.fria.nu/artikel/88431}{\emph{Fria tidningen}} 2011-05-30.\\

  \noindent \emph{Opinion pieces} \\
  --- 20160402, Svenska Dagbladet (with Niclas Hagen), \href{http://www.svd.se/empati-och-solidaritet-tar-nya-former-pa-natet}{``Empati och solidaritet tar nya former på nätet''}. \\
  --- 20160322, Expressen, \href{http://www.expressen.se/debatt/finns-ingen-anledning-att-cyberkapprusta/}{``Det finns ingen anledning att cyberkapprusta''}. \\
  --- 20151119, Expressen, \href{http://www.expressen.se/debatt/overvakningsfragor-maste-fa-ta-sin-tid/}{``Övervakningsfrågor måste få ta sin tid''}. \\
  --- 20130907, Expressen, \href{http://www.expressen.se/debatt/nu-maste-vi-stanga-bakdorrarna/}{``Nu måste vi stänga bakdörrarna''.} \\
  --- 20130611, Expressen, \href{http://www.expressen.se/debatt/det-stora-avslojandet-aterstar-for-snowden/}{```Det stora avslöjandet återstår för Snowden''}. \\
  --- 20110221, Expressen, \href{http://www.expressen.se/debatt/christopher-kullenberg-akrobatikrekord-i-regeringskonst/}{``Akrobatirekord i regeringskonst''}. \\
  --- 20101118, SVT Debatt, \href{}{``Wikileaks är större än den häktade Julian Assange''}. \\
  --- 20100928, SVT Debatt, \href{}{``Socialdemokraterna bör vara stolta över att ha banat väg för The Pirate Bay''}. \\
  --- 20100907, Svenska Dagbladet, \href{http://www.svd.se/internetoperatorerna-maste-skyddas-fran-upphovsrattsindustrin}{``Internetoperatörerna måste skyddas från upphovsrättsindustrin''}. \\
  --- 20100717, Svenska Dagbladet (with Marcin de Kaminski), \href{http://www.svd.se/internet-ar-redan-trasigt}{``Internet är redan trasigt''}. \\
  --- 20090709, Newsmill, \href{}{``Google borde stå på internetmedborgarnas och inte diktaturernas sida''}. \\
  --- 20090617, Newsmill, \href{}{``Så hjälper The Pirate Bay motståndet i Iran''}. \\
  --- 20090506, Newsmill, \href{}{``Bloggarna står för den bästa rapporteringen kring EU:s viktigaste fråga''}. \\
  --- 20090504, Newsmill, \href{}{``Telekompaketet en medborgarrättslig katastrof''}. \\
  --- 20090429, Expressen, \href{http://www.expressen.se/debatt/sverige-maste-raddas-undan-var-internetfientliga-regering/}{``Sverige måste räddas undan vår internetfientliga regering''}. \\
  --- 20081219, Svenska Dagbladet, \href{}{``Nya politiska idéer smittar som virus''}. \\
  --- 20081113, Sydsvenskan, \href{http://www.sydsvenskan.se/2008-11-12/piratjagarlagen}{``Piratjägarlagen''}. \\
  --- 20081007, Svenska Dagbladet, \href{http://www.svd.se/det-digitala-livet-ar-inte-mindre-verkligt}{``Det digitala livet är inte mindre verkligt''}. \\
  --- 20080926, Expressen, \href{}{``De förtjänar inte kallas liberaler''}. \\
  --- 20080916, Expressen (with Mark Klamberg \& Karl Palmås), \href{http://www.expressen.se/kultur/silence-fiction/}{``Silence Fiction''}. \\
  --- 20080916, Aftonbladet (with Emina Karic), \href{}{``FRA:s signalspaning liknar Sovjetssystemets centraldator''}. \\
  --- 20080916, Sydsvenskan, \href{}{``Kreativitet bästa motståndet mot FRA''}. \\
  --- 20080715, Expressen, (with Evelina Wahlqvist), \href{http://www.expressen.se/debatt/ryssland-ar-inte-hotet-bergling/}{``Ryssland är inte hotet Bergling!''}. \\
  --- 20080709, Svenska Dagbladet, \href{http://www.svd.se/fra-kan-visst}{``FRA kan visst!''}. \\
  --- 20080704, Aftonbladet, \href{}{``Anti-FRA – den nya liberalismen''}. \\
  --- 20080616, Expressen, \href{http://www.expressen.se/debatt/fra-lagen-hindrar-fria-forskningen/}{``FRA hotar den fria forskningen''}. \\
  --- 20080616, Svenska Dagbladet, \href{http://www.svd.se/meddelarskyddet-bort-med-ett-knapptryck}{``Meddelarskyddet borta med ett knapptryck''}. \\
  --- 20061011, Svenska Dagbladet, \href{http://www.svd.se/bodstrom-forlorar-snart-natkontrollen}{``Bodström förlorar snart nätkontrollen''}. \\
  --- 20060806, Svenska Dagbladet,  \href{http://www.svd.se/vi-har-inget-att-oroa-oss-for}{``Vi har inget att oroa oss för''}. \\



   \noindent \emph{Popularizations: book chapters and articles}\\
     --- (2017, in press) ``Transistorrytm'', \emph{Monad}, No. 2 (2017). \\
     --- (2016) ``Alla kan forska - Vad är medborgarvetenskap och hur skiljer den sig från vanlig vetenskap'', \emph{Modern filosofi}, 2016(3), 68-69. \\
     --- (2013) ``Nätets Geopolitik'', in \emph{Det är vår värld – tio unga röster om global solidaritet}, Stockholm: A-smedjan.\\
     --- (2010) ``Teknik är samhället gjort hållbart… eller?'', in Tovhult Klara (eds) \emph{Kunskap, kommunikation, kontroll – Drömmar och farhågor i informationssamhället}, Stockholm: Sharing is Caring Förlag. \\

    \noindent \emph{Collections}\\
      ---, \& Lehne, J. (2009) \emph{Resistance Studies Reader 2008}, London and Gothenburg: Resistance Studies Network.\\

    \noindent \emph{Open data repositories}\\
    --- (2015) \& Nelhans, Gustaf. (2015). The happiness turn? Mapping the
    Emergence of “Happiness Studies” Using Cited References. Version 1.0. Svensk
    Nationell Datatjänst. \href{http://dx.doi.org/10.5878/002633}{DOI 10.5878/002633}.
    \\

    \noindent \emph{Software}\\
    --- (2015) \emph{Cite.py}, Citation analysis software, \href{https://github.com/christopherkullenberg/Citepy}{github.com/christopherkullenberg/Citepy}.\\
    --- (2016) \href{http://offentligautredningar.flov.gu.se/}{offentligautredningar.flov.gu.se}, Search engine for Statens offentliga utredningar,\\ \href{https://github.com/christopherkullenberg/offentligautredningar.se}{github.com/christopherkullenberg/offentligautredningar.se}. \\
    --- (2016) \href{https://genuskollen.se}{Genuskollen.se}, Gender counter algorithm based on names,\\ \href{https://github.com/christopherkullenberg/gendercounter}{github.com/christopherkullenberg/gendercounter}. \\


%%%%%%%%%%%%%%%%%%%%%%%%%%%%%%%%%%%%%%%%%%%%%%%%%%%%%%%
%%%%%%% RESEARCH PROJECTS %%%%%%%%%%%%%%%%%%%%%%%%%%%%%
%%%%%%%%%%%%%%%%%%%%%%%%%%%%%%%%%%%%%%%%%%%%%%%%%%%%%%%

%    2. Forskningsprojekt och forskningssamarbeten
%    a) externfinansierade
%    b) ledning av projekt
%    c) medverkan i projekt
%    d) andra former av forskningssamarbeten (nationellt/internationellt)
\subsection{Research Projects \& Collaborations}

    \subsubsection{Externally funded}
    \begin{itemize}
      \item Researcher 50\%, 2014-2018, \textbf{co-applicant.} \\
      ``Taking Science to the Crowd: Researchers, Programmers and Volunteer  \\
      Contributors Transforming Science Online.''\\ Marianne \& Marcus Wallenberg Foundation\\
      PI: \href{mailto:dick.kasperowski@gu.se}{Dick Kasperowski}, Grant number: \href{https://www.wallenberg.com/MMW/projektanslag-2013}{MMW 2013.0020}.
      \item Researcher 50\%, 2014-2016, \textbf{co-applicant.} \\
      ``The Co-production of Social Science and Society: The Case of Happiness studies.''  \\
      Swedish Research Council (Vetenskapsrådet)\\
      PI: \href{mailto:margareta.hallberg@gu.se}{Margareta Hallberg}, Grant number: \href{http://vrproj.vr.se/detail.asp?arendeid=90421}{2012-1117}.
      \item Researcher 50\%, October 2012 – March 2013, \textbf{main applicant.}\\
      ``Subcultures on the Net: Resistance and Engagement in Knowledge Practices''\\
      LETStudio, pilot project.
    \end{itemize}

    \subsubsection{Research collaborations}
    \begin{itemize}
      \item Member of the \href{https://citizenscienceassociation.org/overview/steering-committees/#metadata}{Citizen Science Association Working Group for Data and Metadata.}
      \item Member of the \href{http://letstudio.gu.se/members/christopher-kullenberg}{Learning and Media Technology Studio, LETStudio, University of Gothenburg.}
      \item Board Member of the \href{http://cassirer.se/sallskapet/styrelse/}{Swedish Ernst Cassirer Society}.
    \end{itemize}



%%%%%%%%%%%%%%%%%%%%%%%%%%%%%%%%%%%%%%%%%%%%%%%%%%%%%%%
%%%%%%%%% REFEREE %%%%%%%%%%%%%%%%%%%%%%%%%%%%%%%%%%%%%
%%%%%%%%%%%%%%%%%%%%%%%%%%%%%%%%%%%%%%%%%%%%%%%%%%%%%%%

%3. Vetenskapliga bedömningsuppdrag – nationella och internationella
%a) sakkunnig vid anställning eller för befordran
%b) opponentskap
%c) betygsnämndsledamot
%d) referee på vetenskapliga arbeten
%e) sakkunnig på forskningsansökningar
%f) övrigt

\subsection{Referee work}
    % Lägg till b) opponentskap att jag varit opponent på Eriks slutseminarium

    \subsubsection{Pre-publication reviews}
    (see also: \href{publons.com/a/1172130/}{publons.com/a/1172130/})\\
    - 2017. \emph{Lychnos}.\\
    - 2016. \emph{PLOS ONE}. \\
    - 2016. \emph{Nordic Journal of Science and Technology Studies}.\\
    - 2015. \emph{Journal of the Royal Society of New Zealand}. \\
    - 2015. \emph{Higher Education}.\\
    - 2015. \emph{Organization Studies}.\\
    - 2015. \emph{Journal of Resistance Studies}.\\
    - 2015. \emph{Journal of Resistance Studies}. \\

    \subsubsection{Research applications reviews}
    \begin{itemize}
      \item 2017. \href{http://files.christopherkullenberg.se/NWOreview.pdf}{Netherlands Organisation for Scientific Research (NWO), Veni Research Proposal}.
      \item \href{http://files.christopherkullenberg.se/erc.pdf}{2015. ERC Consolidator Grant Call 2015, European Research Council}.
    \end{itemize}

    \subsubsection{Editorial board membership}
    \begin{itemize}
      \item \href{http://resistance-journal.org/the-old-rs-mag/}{2008 - 2010. Editor \& founder, Resistance Studies Magazine.}
      \item \href{http://resistance-journal.org/editorialboard/}{2015 - present, Editorial board member, Resistance Studies Journal.}
    \end{itemize}


%%%%%%%%%%%%%%%%%%%%%%%%%%%%%%%%%%%%%%%%%%%%%%%%%%%%%%%
%%%%%%%%%%% OTHER MERITS %%%%%%%%%%%%%%%%%%%%%%%%%%%%%%
%%%%%%%%%%%%%%%%%%%%%%%%%%%%%%%%%%%%%%%%%%%%%%%%%%%%%%%
      %4. Andra uppgifter i anslutning till forskning på universitets-, fakultets- eller insti-
      %tutionsnivå
      %5. Övriga meriter
      %a) organisation av konferenser
      %b) konferensinbjudan (plenarföredrag)
      %c) konferensdeltagande (papers, posters)
      %d) gästforskare vid utländskt lärosäte
      %e) övriga inbjudningar
      %f) priser och andra utmärkelser
      %g) redaktörskap/redaktionskommitté för tidskrift
      %h) andra meriter

%\subsection{Conference organization}

%%%%%%%%%%%%%%%%%%%%%%%%%%%%%%%%%%%%%%%%%%%%%%%%%%%%%%%
%%%%%%%%%%%%%% PUBLIC OUTREACH %%%%%%%%%%%%%%%%%%%%%%%%
%%%%%%%%%%%%%%%%%%%%%%%%%%%%%%%%%%%%%%%%%%%%%%%%%%%%%%%

\subsection{Public outreach and extra-academic collaborations}

    \subsubsection{Interviews in mass media}
    % Universitetsläraren om citizen science
    % - 20160914 - SR P4 Västernorrland ()
    % kolla upp Universitetsläraren sulftidningen

    % P1 vetenskapsradion om anslagstavlan..
   %- 20170522 - SR P4 Gävleborg.
    - 20170503 - SVT Nyheter, \href{https://www.svt.se/nyheter/lokalt/vast/elever-har-kartlagt-anslagstavlor-i-hela-landet}{Kartläggning: Detta står på anslagstavlor}. \\
    - 20170503 - SR P4 Östergötland, \href{http://t.sr.se/2pdhEBb}{Forskare: Inbjudningar vanligast på anslagstavlor}.\\
    - 20170117 - Curie, \href{http://www.tidningencurie.se/nyheter/2017/01/17/trycket-okar-gor-forskningsdata-tillgangliga/}{Trycket ökar: Gör forskningsdata tillgängliga}.\\
    - 20160007 - GU-journalen 5-2016, \href{https://issuu.com/universityofgothenburg/docs/gu-journalen5-2016/34}{``Anslagstavlan''}.\\
    - 20161008 - Dagens Nyheter, \href{http://www.dn.se/nyheter/sverige/synen-pa-oppen-publicering-delar-forskarna/}{``Synen på öppen publicering delar forskarna''}.\\
    - 20161005 - Sveriges Radio, Kulturnytt, \href{http://sverigesradio.se/sida/artikel.aspx?programid=478&artikel=6533649}{``Facebooks balansgång mellan globala krav''}.\\
    - 20161002 - Arbetarbladet, \href{http://www.arbetarbladet.se/gavleborg/sandviken/vilken-betydelse-har-anslagstavlan-i-dag}{``Vilken betydelse har anslagstavlan idag?}.\\
    - 20160930 - SR P1, \href{http://t.sr.se/2dszL0a}{Vetenskapsradions veckomagasin}.\\
    - 20160930 - GU-Journalen, \href{https://issuu.com/universityofgothenburg/docs/guj4-2016}{``Macchiarini did not operate in a vacuum''}.\\
    - 20160929 - Dagens Samhälle, ``Elever med mobilapp fångar analoga anslag''.\\
    - 20160811 - DagensETC, ``Forskningen har flyttat in i hemmet''. \\
    - 20160915 - Språktidningen, \href{http://spraktidningen.se/blogg/skolelever-far-prova-pa-forskarlivet}{``Skolelever får pröva på forskarlivet}.\\
    - 20160530 - P4 Östergötland, \href{http://t.sr.se/1NY9nMZ}{``Landets anslagstavlor ska kartläggas''}. \\
    - 20160523 - Piteåtidningen – \href{http://www.pt.se/nyheter/pitea/elever-ska-fota-anslagstavlor-10045575.aspx}{``Elever ska fota anslagstavlor''}. \\
    - 20160316 - Svenska Dagbladet, \href{http://www.svd.se/sanning-inte-viktigt-for-ett-drev/om/natdrev}{``Sanningen inte viktig för ett drev''}. \\
    - 20160222 - Sveriges Radio - P1 Morgon: \href{http://t.sr.se/1KBHSak}{``Varför vägrar Apple hacka en telefon åt FBI''}. \\
    - 20160218 - Götheborske Spionen, \href{http://www.spionen.se/140-redaktionellt/reportage/feature/1593-sa-oevervakas-du-via-naetet}{``Så övervakas du via nätet''}. \\
    - 20160217 - Sesam, ``Vanliga människor hjälper forskare''.\\
    - 20160211 - Fria tidningen, \href{http://www.fria.nu/artikel/121724}{``Allmänheten är den nya forskaren''}. \\
    - 20160115 - GU-journalen, \href{https://issuu.com/universityofgothenburg/docs/guj1-2016/32}{``Vi behöver en lag för internet''}. \\
    - 20160118 - SR P1 Vetenskapsradion, \href{http://t.sr.se/239vggG}{Här rycker medborgarforskaren in.}\\
    - 20160118 - Miljöaktuellt, \href{http://miljoaktuellt.se/vanligare-att-forskare-tar-hjalp-av-allmanheten/}{``Vanligt med medborgarforskning på miljöområdet''}. \\
    - 20160119 - Discover Magazine, \href{http://blogs.discovermagazine.com/inkfish/2016/01/19/what-is-citizen-science-good-for-birds-butterflies-big-data/#.Vp9DvPGFB24}{``What Is Citizen Science Good For? Birds, Butterflies, Big Data''}. \\
    - 20160120 - Journal of Science Communication, \href{http://jcom.sissa.it/node/3076}{``New study on Citizen Science’s impact on scientific research''}. \\
    - 20151207 - Sveriges Radio - P1 Morgon, \href{http://t.sr.se/1XMYemi}{``Vad innebär de nya åtgärderna mot terrorism?''}. \\
    - 20150729 - ARTE, Tracks, \href{http://tracks.arte.tv/fr/christopher-kullenberg-world-war-web}{``Christopher Kullenberg, World War Web''.} \\
    - 20150601 - Hallands Nyheter, ``Kommunen betalar med informationen''.\\
    - 20150323 - Sveriges Television Vetenskapens Värld, \href{https://youtu.be/vG8sZQnU7mU?t=18m1s}{``Medborgarforskning.''} \\
    - 20150103 - SVT Nyheter, \href{http://www.svt.se/nyheter/utrikes/gmail-senaste-malet-for-kinas-natcensur?cmpid=del:pd:ny:20160803:gmail-senaste-malet-for-kinas-natcensur:nyh}{``Gmail senaste offret för Kinas nätcensur''}. \\
    %- 20140415 \href{http://sverigesradio.se/sida/artikel.aspx?programid=1646&amp;artikel=5837822}{SR P3, Är lagstiftning rätt väg för att stoppa näthatet?}. \\
    - 20141108 - Sydsvenskan, ``Anonymitet hotad för knarkköpare''. \\
    - 20140329 - Sydsvenskan, \href{http://www.sydsvenskan.se/2014-03-29/okad-avlyssning-kraver-nya-motvapen}{``Ökad avlyssning kräver nya motvapen''}.\\
    - 20131021 - SR P4, \href{http://t.sr.se/NC8sUF}{``Övervakningens nödutgångar''}. \\
    - 20130912 - Aftonbladet, \href{http://www.aftonbladet.se/nyheter/vetmer/article17455373.ab}{``Ditt finger i deras händer''}. \\
    - 20130907 - Information.dk, \href{http://www.information.dk/471440}{``Det er godt, at vi smadrer myten om det frie internet''}. \\
    - 20130906 - Sveriges Radio P1, \href{http://t.sr.se/OfP0gG}{``Hur trygga är vi på nätet?''} \\
    - 20130719 - Dagens Nyheter, \href{http://www.dn.se/kultur-noje/sjalvforstorande-bilder-gor-succe/}{``Självförstörande bilder gör succe''}. \\
    - 20130615 - Helsingborgs Dagblad, \href{http://www.hd.se/2013-06-15/storebror-ser-dig---men-bryr-vi-oss}{``Storebror ser dig, men bryr vi oss?''} \\
    - 20130613 - Arbetaren, ``Obama-administrationen måste förklara sig''. \\
    - 20130607 - Sverige Radio P3, \href{http://t.sr.se/MiZb35}{``Korrerapporten.''} \\
    - 20130601 - Utbildningsradion, \href{http://urskola.se/Produkter/176858-UR-Samtiden-Overvakning-och-kontroll-Internetovervakning-och-revolutioner}{``Internetövervakning och revolutioner''.} \\
    - 20130217 - Sydsvenskan, \href{http://www.sydsvenskan.se/2013-02-17/tecken-i-tiden}{``Tecken i tiden''}.\\
    - 20130214 - Internetworld, \href{http://www.idg.se/2.1085/1.492033/han-driver-pa-for-ett-oppet-nat}{``Han driver på för ett öppet nät.''}. \\
    - 20121004 - Sveriges Radio Studio ett, \href{http://t.sr.se/1foOdCK}{``Simulerade och verkliga nätattacker.''} \\
    - 20120430 - Sveriges Radio Studio ett, \href{http://t.sr.se/1cRSOfk}{``Kan man lita på Wikipedia?''} \\
    - 20120321 - Nerikes Allehanda, \href{http://na.se/nyheter/orebro/1.1589481--de-som-inte-vill-blir-inte-upptackta-}{``De som inte vill bli upptäckta''}. \\
    - 20120306 - Internetworld, \href{http://www.idg.se/2.1085/1.435933/kreativiteten-ar-inte-individuell}{``Kreativiteten är inte individuell''}. \\
    %- 20120301 - Akademisk kvart, \href{http://www.youtube.com/watch?v=-HYfVmanye8}{``Humanistisk forskning och engagemang''}. \\
    - 20111226 - Forbes, \href{http://www.forbes.com/sites/andygreenberg/2011/12/26/meet-telecomix-the-hackers-bent-on-exposing-those-who-censor-and-surveil-the-internet/}{``Meet Telecomix, The Hackers Bent On Exposing Those Who Censor And Surveil The Internet''}. \\
    - 20111204 - SR P4 Jönköping, \href{http://t.sr.se/1gy095E}{``Christopher från Bodafors är Årets svensk.''} \\
    - 2011 - ARTE, TRACKS, ``Christopher Kullenberg'', Also on Youtube: \href{https://www.youtube.com/watch?v=a3ppeRZHVcY}{https://www.youtube.com/watch?v=a3ppeRZHVcY}.\\
    - 20111206 - Voxeurop, \href{http://www.voxeurop.eu/en/content/article/1254651-cyber-revolutionary-tahrir-square}{``The cyber-revolutionary on Tahrir Square''}. \\
    - 20111206 - Washington Post, \href{https://www.washingtonpost.com/lifestyle/style/the-hacktivists-of-telecomix-lend-a-hand-to-the-arab-spring/2011/12/05/gIQAAosraO_story.html}{``The ‘hacktivists’ of Telecomix lend a hand to the Arab Spring''}. \\
    - 20111201 - Fokus, \href{http://www.fokus.se/2011/12/med-datorn-som-vapen/}{``Med datorn som vapen''}. \\
    %- 20111108 - Dagens Nyheter, \href{http://humaninternet.info/intervjuwl.jpg}{Dagens Nyheter}. \\
    %- 20111023 - FN-dagen, {``Sociala medier och den arabiska våren''}. \\
    - 20110811 - Wired.it, \href{http://daily.wired.it/news/internet/2011/08/11/telecomix-hacker-egitto-tunisia-iran-13861.html}{``Telecomix, gli hacker di una volta''}. \\
    %- 20110810 - SR Sahlberg i P1 - “Den Kinesiska Nätcensuren”. Download as \href{http://humaninternet.info/SRKina20110810.ogg}{.ogg}. \\
    %- 20110626 - SR Godmorgon Världen \href{http://humaninternet.info/SR_GV_20110626.mp3}{mp3}. \\
    %- 20110611 - Brunchrapporten (P3):\href{http://sverigesradio.se/topsy/ljudfil/3187937.mp3}{whole program (MP3)}. \\
    - 20110607 - Computer Sweden, \href{http://computersweden.idg.se/2.2683/1.388901/han-praktiserar-yttrandefrihet}{``Han praktiserar yttrandefrihet''}. \\
    - 20110606 - Dagens Nyheter, \href{http://www.dn.se/nyheter/varlden/sa-begransar-syrien-friheten-pa-internet}{``Så begränsar Syrien internet.''} \\
    %- 20110526 - SR P3 Brunchrapporten \href{http://humaninternet.info/brunchrapporten20110526.ogg}{OGG direct download} \\
    - 20110407 - SR P1 Obs!, \href{http://t.sr.se/1zjt3PQ}{``Internet, Gud och människan''}. \\
    - 20110407 - SR P1, \href{http://t.sr.se/PbptWr}{``Internet, ett nytt politiskt subjekt?''} \\
    - 20110407 - Hackerspaces Signal Radio, \href{http://signal.hackerspaces.org/archive/2011-04-07-2200-hacktivism-hour.mp3}{``Hacktivism hour''}. \\
    - 20110322 - Jusektidningen, \href{http://www.tidningenkarriar.se/Arkivet/2011/3/Anarkistisk-hjalte/}{``En anarkistisk hjälte''}. \\
    %- 20110309 \href{http://sverigesradio.se/sida/artikel.aspx?programid=2422&amp;artikel=4390966}{SR P4 Göteborg}. \\
    %- 20110308 \href{http://humaninternet.info/S146_hacktivistes_web.ogv}{.ogv}. \\
    - 20110308 - Dagens Nyheter, \href{http://www.dn.se/vart-internet/vart-internet-hem/sa-kan-stater-hota-friheten-pa-natet/}{``Så kan stater hota friheten på nätet''}. \\
    - 20110224 - Dagens Nyheter, \href{http://www.dn.se/vart-internet/vart-internet-hem/din-chatt-en-sakerhetsrisk/}{``Din chatt - en säkerhetsrisk''}. \\
    - 20110221 - Sveriges Radio Studio ett, \href{http://t.sr.se/1cUlHfI}{``Studio Ett- extrasändning om Libyen.''} \\
    - 20110214 - Sveriges Radio Studio ett, \href{http://t.sr.se/1D1zCXq}{``Iran - oppositionen uppmanar till demonstrationer.''} \\
    - 20110202 - PC-facile, \href{http://www.pc-facile.com/news/come_comunica_egitto_senza_internet/68959.htm}{``Come comunica l'Egitto senza Internet''}. \\
    - 20110201 - New Scientist, \href{http://www.newscientist.com/blogs/onepercent/2011/02/egypt-remains-officially-offli.html}{``How Egypt is getting online without the internet''}. \\
    - 20110131 - Deutsche Welle, \href{http://www.dw.com/en/european-activists-offer-dial-up-internet-to-get-egypt-back-online/a-14807049}{``European activists offer dial-up Internet to get Egypt back online''}. \\
    - 20110131 - Radio Netherlands, \href{http://www.rnw.nl/english/article/dust-your-dialup-modem-contact-egypt}{``Dust off your old modem for Egypt''}. \\
    - 20110131 - Huffington Post,  \href{http://www.huffingtonpost.com/2011/01/29/anonymous-internet-egypt_n_815889.html}{``Anonymous Internet Users Team Up To Provide Communication Tools For Egyptian People''}. \\
    - 20110129 - Sveriges Television, \href{http://www.youtube.com/watch?v=HRZ0QKyiFsE&amp;feature=related}{``Rapport''.} \\
    %- 20110129 \href{http://sverigesradio.se/sida/artikel.aspx?programid=1300&amp;artikel=4319039}{SR P1 Konflikt}. \\
    %- 20110128 - SR P1 \href{http://humaninternet.info/srp1egypten.ogg}{ogg}. \\
    %- 20110128 - YLE Radio Vega, (at 38:40 in this \href{http://humaninternet.info/YLE20110128.ogg}{ogg}. \\
    %- 20110126 - SR Obs!, \href{http://static.sr.se/laddahem/podradio/SR_p1_obs_110126020100.mp3}{this show (mp3)}. \\
    - 20110114 - Hackerspaces Signal Radio, \href{http://signal.hackerspaces.org/archive/2011-01-13-2200-hacktivism-hour.mp3}{``Hackerspaces Signal Radio''}. \\
    - 20110117 - Svenska Dagbladet,  \href{http://www.svd.se/stjarnlackan-far-konkurrens}{``Stjärnläckan får konkurrens''}. \\
    - 20101220 - Hufvudstadsbladet, ``En rörelse som inte går att stoppa''.\\
    %- 20101211 - SR P3 Brunchrapporten, \href{http://humaninternet.info/SR_p3_brunch_101209013544.mp3}{MP3}. \\
    - 20101206 - SR P1 Studio ett, \href{http://t.sr.se/1CQkYCj}{``Nätfriheten hotas''}. \\
    - 20101115 - SR P3 Brunchrapporten, \href{http://t.sr.se/RI5PTm}{``Cyberkrig, digital atombomb och hackersoldater''}. \\
    - 20101017 - Sydsvenskan, \href{http://www.sydsvenskan.se/2010-10-17/trollen-harskar-i-natets-utmarker}{``Trollen härskar i nätets utmarker''}. \\
    - 20100611 - NRLI TV @ Hacknight, \href{http://www.youtube.com/watch?v=VhmH6cJZS4k&amp;feature=related}{``How to Hack Politics. Part II Interview with Christopher Kullenberg''}. \\
    - 20100601 - Sveriges Radio P3 Brunchrapporten, \href{http://t.sr.se/1cnEijY}{``Kommunikationsstopp för Ship to Gaza''.} \\
    - 20090729 - Dagens Nyheter, \href{http://www.dn.se/kultur-noje/nyheter/internatet-som-vill-bygga-om-eu/}{``Internätet som vill bygga om EU''}. \\
    - 20090427 - Peppar.fi, ``EU Röstar om internets framtid''. \\
    - 20090422 - Dagens Nyheter, \href{http://www.dn.se/kultur-noje/musik/for-tidigt-ropa-hej-internet-inte-raddat/}{``För tidigt ropa hej - internet inte räddat''}. \\
    - 20090416 - Dagens Nyheter \href{http://www.dn.se/kultur-noje/en-krigsforklaring-mot-internet/}{``En krigsförklaring mot internet''}. \\
    - 20081215 - GU-Journalen, \href{http://www.gu-journalen.gu.se/english/News/News_detail/?contentId=855527}{``Journal across boundaries''}. \\
    - 20081219 - Sheffield Live! Radio, \href{http://www.dcs.shef.ac.uk/%7Enoel/soundofscience/Prog%2032.mp3}{``Sound of Science''}. \\
    - 20080626 - Proletären, \href{http://www.proletaren.se/inrikes/%E2%80%9Dfra-lagen-hotar-kritisk-forskning}{``FRA-lagen hotar kritisk forskning''}.

    \subsubsection{Invited speaker}
    - 20170426 - \href{https://ecsa.citizen-science.net/sites/default/files/draft_agenda_second_workshop_appsplatforms.pdf}{``Collecting Social Science Data with Smartphone Apps and School Children''}, ECSA workshop ``Defining Principles for Mobile Apps and Platforms Development in Citizen Science'', Gothenburg, Sweden. \\
    - 20170330 - ``Vad händer om vi inte kan enas om vad som är sant?'', \href{http://flov.gu.se/aktuellt/Nyheter/fulltext//oppet-hus-pa-flov-med-panelsamtal-om-alternativa-fakta-och-informationsbubblor.cid1428426}{FLoV, Göteborgs Universitet}.\\
    - 20170329 - ``Vad är på gång inom citizen science?'', \href{http://www.vinnova.se/sv/Aktuellt--publicerat/Kalendarium/2017/170329-Informationsmote-Vetenskap-med-och-for-samhallet/}{Informationsmöte Vetenskap med och för samhället Horisont 2020}, Vinnova, Stockholm. Lecture available on \href{https://youtu.be/HHW0j7Zo5E0}{Youtube}.\\
    - 20161214 - ``Digital methods - Theory, Practice and Ethics'', Higher seminar, Karlstad University.\\
    - 20161116 - \href{http://www.stadsbiblioteket.nu/tusen-plataer/}{``Tusen platåer''}, Göteborg, Stadsbiblioteket.\\
    - 20161028 - ``Statens offentliga utredningar digitaliserade'', Högre seminarium, Idéhistoria, Lunds universitet.\\
    - 20161012 - \href{https://v-a.se/events/va-dagen-2016/}{``VA-Dagen 2016 om öppen vetenskap''}, Stockholm. \\Video: \href{https://youtu.be/CUNnddKvZ9o?t=43m52s}{https://youtu.be/CUNnddKvZ9o?t=43m52s}.\\
    - 20160922 - \href{http://flov.gu.se/aktuellt/Nyheter/fulltext//sju-filosofer-forelaser-pa-bokmassan-.cid1403852}{``Kan alla forska?''}, Forskartorget, Bokmässan, Göteborg. \\
    - 20160921 - \href{http://kompetensutveckling.adm.gu.se/seminar/detail/2260}{``Att nå ut via sociala medier''}, University of Gothenburg. \\
    - 20160912 - \href{http://www.kb.se/aktuellt/utbildningar/2016/Kulturarvet-som-ettor-och-nollor--Del-3-Digital-humaniora/}{``Kulturavet som ettor och nollor''}, National Library of Sweden.\\
    - 20160209 - \href{http://www.abfgoteborg.org/index.php/archive/2016/170-filosofiscenen-2016/2208-sarah-ahmed-lycka-och-falskt-medvetande}{``Sara Ahmed - lycka, olycka och falskt medvetande''}, ABF, Göteborg, Sweden.\\
    - 20150929 - \href{http://hum.gu.se/aktuellt/Nyheter/fulltext//nackrostimmen--spionen-i-fickan-som-overvakar-oss.cid1324880}{``Spionen i Fickan''}, Näckrostimmen, University of Gothenburg.\\
    - 20151209 - \href{http://fhp.nu/tusenplataer}{``Samtal om Tusen Platåer''}, Bio Rio, Stockholm, Sweden.\\
    - 20151020 - \href{http://www.kb.se/aktuellt/evenemang/2015/SOUhack/}{``Statens röst digitaliserad''}, National Library of Sweden. \\
    - 20141126 - \href{http://www.wherevent.com/detail/Goteborgs-Konsthall-TUNNELPOLITIK-Forelasning-med-Christopher-Kullenberg}{Tunnelpolitik}, Konsthallen, Gothenburg.
    - 20141122 - \href{https://www.facebook.com/events/622473101197803/}{``Re-building a common internet''}, Verkko Suljettu, Helsinki, Finland. \\
    - 20140604 - \href{http://www.urbsec.se/digitalAssets/1483/1483366_program-urbsec-konferens-2014-06-04.pdf}{``Forskning, allmänhet, innovation – Crowd Science som nytt forskningsparadigm'', URBSEC, Gothenburg, Sweden.} \\
    - 20130223 - \href{http://www.frilansjournalisten.nu/2013/01/sasong-for-arsmoten/}{``Kommunikation och nätet''}, Smålands Frilansklubb, Växjö, Sweden. \\
    - 20121207 - \href{http://letstudio.gu.se/svenska/aktuellt/nyheter/n/christopher-kullenberg-inbjuden-som-talare-pa-internationell-konferens.cid1111971}{``Does Internet freedom have a price? Examples from the Arab Spring''}, \href{http://www.hre2012.uj.edu.pl/invited-speakers}{The Third International Conference on Human Rights Education, Jagiellonian University, Krakow, Poland.} \\
    - 20121105 - \href{https://www.youtube.com/watch?v=Zo24Qy_PU8I}{``Vilken roll spelar de nya internetbaserade medierna i samhällsförändrande aktiviteter?''}, Sigmas inspirationsdag, Gothenburg, Sweden. \\
    - 20120822 - \href{https://transmediale.de/content/resource-002-out-place-out-time}{``reSource 002: Out of Place, Out of Time''}, Transmediale, Berlin, Germany. \\
    - 20120509 - \href{http://pol.gu.se/aktuellt/kalendarium/aktuellt_detalj/?eventId=1777397828}{''The Arabic Spring. Internet and social media - crucial tools in organizing the civil uprisings in Tunis and Egypt.''}, Brännpunkt Europa, University of Gothenburg. \\
    - 20120301 - \href{http://www.youtube.com/watch?v=-HYfVmanye8}{``Humanistisk forskning och engagemang''}, Akademisk kvart, University of Gothenburg, Sweden. \\
    - 20111124 - \href{https://www.youtube.com/watch?v=zWROWpMaKmE}{``Our Internet - Our Rights, Our Freedoms''}, Council of Europe, Vienna, Austria. \\
    - 20111119 - ``What will be the needs of tomorrow's leaders'', Swedish Embassy Cairo, Cairo, Egypt. \\
    - 20111022 - \href{http://www.youtube.com/watch?v=6Mi0g93ModU}{``Nätaktivism – nya möjligheter men också faror''}. Svenska FN-förbundet, Göteborg, Sweden. \\
    - 20111304 - \href{http://www.rsf-ch.ch/anonymat-des-communications-s%C3%A9curit%C3%A9-des-donn%C3%A9es-protection-des-sources-0}{``Anonymat des communications, sécurité des données, protection des sources''}, Reporters Sans Frontieres, Geneva, Switzerland. \\
    - 20100504 - ``Greens/EFA and Activists Workshop on ACTA and the Public interest'', European Parliament, Brussels, Belgium. \\
    - 20090604 - \href{https://issuu.com/deutsche-welle/docs/program-deutsche-welle-global-media-forum-2009}{``Information Technology: Provoking or Preventing Conflict''}, Deutsche Welle Global Media Forum, Bonn, Germany. \\
    - 20091107 - \href{https://vimeo.com/10286077}{``Net Neutrality, Surveillance and how to Re-build Politics''}, Free Society Conference, Gothenburg, Sweden. \\
    - 20091103 - \href{https://internetdagarna.se/arkiv/2009/program-2009/3-november.html}{``Revision av telekompaketet''}, Internetdagarna, Stockholm, Sweden. \\
    - 20091028 - ``Invited speaker by the Ministry of Enterprise, Energy and Communications on the implementation of the Telecoms Package'', Government of Sweden, Stockholm, Sweden.\\
    - 20090111 - \href{https://www.youtube.com/watch?v=G9cXIKvywvs}{``Lightning talk: Resistance Studies Magazine''}, 25th Chaos Communication Congress, Berlin, Germany.\\
    - 20081108 - \href{http://www.fiff.de/veranstaltungen/fiff-jahrestagungen/fiff-jahrestagung-2008-krieg-und-frieden-digital/Programmheft.pdf/at_download/file}{``The social impact of IT''}, Krieg und Frieden Digital, Aachen, Germany.



%___________                  .__    .__
%\__    ___/___ _____    ____ |  |__ |__| ____    ____
%  |    |_/ __ \\__  \ _/ ___\|  |  \|  |/    \  / ___\
%  |    |\  ___/ / __ \\  \___|   Y  \  |   |  \/ /_/  >
%  |____| \___  >____  /\___  >___|  /__|___|  /\___  /
%             \/     \/     \/     \/        \//_____/

% U N D E R V I S N I N G
\clearpage
\setlength\parindent{0cm}

\section{Teaching portfolio}

  \subsection{Overview}
  For the past decade I have instructed both disciplinary and interdisciplinary courses
  in Theory of Science and Media \& Communication Studies, on all levels of education.
  I have supervised 23 completed bachelor- and master's theses and I am currently
  co-supervising one Ph.D-candidate. \\

  As a theorist of science, I have taught students from the natural-,
  medical-, social- and human sciences. I have been the principal
  instructor of the large cross-faculty course  \href{http://files.christopherkullenberg.se/kursplaner/NTH001_Teoretiska_och_historiska_perspektiv_pa%cc%8a_naturvetenskap_10512.pdf}{``Theoretical and Historical Perspectives on Science''},
  which hosts up to 100 students from the natural sciences. During this time I
  developed the course with the purpose of not only of deepening the student's
  knowledge on the sociology and history of science, but I also created excercises
  in peer review to introduce them to the core epistemic practice of science.\\

  I have also taught the core disciplinary course \href{http://files.christopherkullenberg.se/kursplaner/VT2106_Klassisk_vetenskapsteori_13187.pdf}{``Classical Theory of Science''},
  both on basic and advanced levels. \\

  During the spring semester of 2014 I was \href{http://files.christopherkullenberg.se/studierektoronline.pdf}{Study administrator (studierektor)} for the subject of theory of science. \\

  During 2004-2006 I taught courses in Media- \& Communication studies and Strategic Communication
  at the \href{http://files.christopherkullenberg.se/universitywest.pdf}{University West} and supervised bachelor's theses in the same subjects.
  At the University of Gothenburg I have been the principal instructor of the course
  \href{http://files.christopherkullenberg.se/kursplaner/KT2102.pdf}{``Communication in New and Social Media''} and supervised master's theses in communication studies.
  With Dick Kasperowski I developed the course \href{http://files.christopherkullenberg.se/kursplaner/KT2110.pdf}{``Communication of Scientific Knowledge''}.\\



    %1. Egen pedagogisk utbildning
    % - högskolepedagogisk kurs
    % - handledarutbildning (obligatorisk)
    \subsection{Formal education, Teaching in higher education (Högskolepedagogik)}
    \begin{itemize}
      \item \href{http://files.christopherkullenberg.se/hogskolepedagogikonline.pdf}{2006. Teaching and Learning in Higher Education} (\href{http://files.christopherkullenberg.se/PE0940.pdf}{Högskolepedagogik för lärare, PE0940}), 7.5 ECTS.
      \item \href{http://files.christopherkullenberg.se/hogskolepedagogikonline.pdf}{2006. Supervision in Postgraduate Education} (\href{http://files.christopherkullenberg.se/PD0180.pdf}{Högskolepedagogisk handledning, PD0180}), 7.5 ECTS.
      \item 2016. Formal credits for the course \href{http://kursplaner.gu.se/svenska/HPE102.pdf}{HPE102 Teaching and learning in Higher Education 2: Discipline Specific Pedagogic}.
    \end{itemize}


    \subsection{Courses \& Thesis supervision, graduate level}
    %2. Undervisning och handledning
    %a) på grund och/eller avancerad nivå
    % - bredd och omfattning
    % - handledning
    % - kursutveckling
    % - tvärvetenskapligt samarbete
    % - övrig

        \subsubsection{Theory of Science}
        \noindent \emph{Course administrator and teacher:}
        \begin{itemize}
          \item 2013. \href{http://files.christopherkullenberg.se/kursplaner/FHV241Strimman.pdf}{``FHV241 The Strand: People, Knowledge and Public Health Practice''}, 6 ECTS, (Strimman, moment 1: Tre konstruktioner av människan och folkhälsoarbete i praktiken), 1.5 ECTS, Second cycle, University of Gothenburg.
          \item 2007 - 2014. (Course administrator 2012 - 2013). \href{http://files.christopherkullenberg.se/kursplaner/NTH001_Teoretiska_och_historiska_perspektiv_pa%cc%8a_naturvetenskap_10512.pdf}{``NTH001 Theoretical and Historical Perspectives on Science, 7.5 ECTS}, First cycle, University of Gothenburg.
          \item 2014. \href{http://files.christopherkullenberg.se/kursplaner/VT1101_Introduktion_i_klassisk_vetenskapsteori__grundkurs_13185.pdf}{``VT1101 Classical Theory of Science, Introductory Course (Introduktion till klassisk vetenskapsteori)''}, 15 ECTS, First cycle, University of Gothenburg.
          \item 2014. \href{http://files.christopherkullenberg.se/kursplaner/VT2106_Klassisk_vetenskapsteori_13187.pdf}{``VT2106 Classical Theory of Science (Klassisk vetenskapsteori)''}, 15 ECTS, Second cycle level, University of Gothenburg.
        \end{itemize}

        \subsubsection{Media \& Communication Studies, Strategic Communication}
             \noindent \emph{Course administrator and teacher:}
             \begin{itemize}
               \item 2012 - 2013. \href{http://files.christopherkullenberg.se/kursplaner/KT2102.pdf}{``Communication in New and Social Media''}, 7.5 ECTS, Second cycle, University of Gothenburg
               \item 2005. ``Media, Individual and Every-day Life'', 7.5 ECTS, Basic level, University West.
               \item 2005. ``Strategic Communication'', 7.5 ECTS, Basic level, University West.
             \end{itemize}

             \noindent \emph{Thesis supervision:} \\
             - Joacim Schmidt (2017) \href{https://gupea.ub.gu.se/bitstream/2077/52545/1/gupea_2077_52545_1.pdf}{``Facebooks möjligheter och risker: En studie om svenska kommuners användande av Facebook''}\\
             - Sofie Andersson (2013) ``Det kommunikativa målarbetet – en kommunal budgetresa från politiker till medarbetare'', University of Gothenburg.\\
             - Malin Arosilta (2013) ``Demokratiska värden som varumärke - En studie om ''Göteborg 2021''.'', University of Gothenburg.\\
             - Anna Cöster (2013) ``Att vara kyrka på internet. Svenska kyrkan i sociala medier.'', University of Gothenburg.\\
             - Charlotta Dahlberg (2013) ``Utbildning till salu – gymnasieskolors kommunikativa strategier. '', University of Gothenburg.\\
             - Jenny Fogelberg (2013) ``Kommunen i sociala medier–transparens och medborgardialog i Varbergs kommun.'', University of Gothenburg.\\
             - Daniel Alexander Jansson (2013) ``Felaktiga Förväntningar – en studie av Arbetsförmedlingens kommunikation mot ungdomar'', University of Gothenburg.\\
             - Sven Lindström (2013) ``Vad håller vi på med? – En kartläggning av kommunikationsprocessen kring kampanjen ”Vad håller du på med?”'', University of Gothenburg.\\
             - Sunanda Malm (2013) `` Målförståelse, kommunikation och varumärkesarbete på en svensk myndighet, sett ur ett medarbetarperspektiv'''', University of Gothenburg.\\
             - Maria Pettersson (2013) ``''Vissa saker är bara jävligt svårt att prata om'' – Om ungas användning av internet för terapeutiska samtal.'', University of Gothenburg.\\
             - Annie Sjölund (2013) ``Skall man skjuta får man ju vårda också'' - En studie om sjuksköterskors uppfattningar om Försvarsmakten'', University of Gothenburg.\\
             - Agneta Slonawski (2013) ``”Jag tror man mest söker information hos kompisar eller på nätet.” - Förutsättningar för att nå unga med information om alkohol och droger för Mini-Maria Göteborg.'', University of Gothenburg.\\
             - Karin Svenner (2013) ``''Synen på kommunikation – en kvalitativ innehållsanalys av femton kommunikationspolicyer i offentlig förvaltning\\
             - Lovisa Vasiliou (2013) ``1.000 bilder av Värmland - En studie av Region Värmlands kommunikationsprocess inför skrivandet av Värmlandsstrategin'', University of Gothenburg.\\
             - Jörgen Wade (2013) ``Förväntningar på socialt intranät och vilken syn på organisationskommunikation dessa förväntningar reflekterar'', University of Gothenburg.\\
             - Charlotta Windeman och Linda Andreasson (2006) ``Meningen måste ju vara att få folk till butiken : en kvalitativ studie av ICA respektive Coop:s profiler och images'',	University West, Media \& Communication studies. \\
             - Andreas Johansson och Ellinor Wetterblad (2006) ``Organisationers medierelationer : en kvantitativ kartläggning av organisationers mediekontakt och medieverktyg'',	University West, Media \& Communication studies. \\
             - Lars Karlsson (2006) ``Primitiv och utan identitet : en kvalitativ analys av synen på ``den andre'' i Metros pratbubbletävling'',	University West, Media \& Communication studies. \\
             - Tobias Hagrenius (2006) ``Samma nyheter i olika tidningar? en kvalitativ innehållsanalys på tre gratistidningar'',	University West, Media \& Communication studies. \\
             %- Susanne Gebauer och Malin Schöldstein (2006) 	University West, Media \& Communication studies. \\
             %- Katarina Clendinning och Malin Lith  (2006)	University West, Media \& Communication studies. \\
             - Jelena Siric, Lorans Zaya (2005) ``Släpp in mig, yao! En studie om den kontroversiella tidningen Gringo'', 	University West, Media \& Communication studies. \\
             - Jeanette Borg, Karolina Colliander (2005) ``Kolla - Kollektivtrafik åt alla'',	University West, Media \& Communication studies. \\
             - Annika Jonasson och Tina Nyrén (2005) ``Terrorism är kommunikation. En diskursanalys av Aftonbladet, Dagens Nyheter och Svenska Dagbladets rapportering av terrordåden vid München 1972, Lockerbie 1988 och World Trade Center 2001'', University West, Media \& Communication studies. \\
             - Daniel Sjöberg (2004) ``Ideologi i amerikanska programprodukter – En receptionsanalys av programmet L-Word baserad på kvalitativa intervjuer med elva respondenter''. University West, Media \& Communication studies. \\
             - Christina Axelsson och Helena Johansson (2004) ``LunarStorm - Framtidens fritidsgård? En kvalitativ studie om unga människors användning av Internetcommunities och deras inställningar till betaltjänster'', University West, Media \& Communication studies. \\

    %b) på forskarnivå
    % handledning av doktorander (skilj mellan huvudhandledare och bi handledare, disputerade och ännu ej disputerade, ange namn och  disputationsår, i de fall handledning varit aktuell under endast en del  av forskarsutbildningen ska tidsomfånget anges )
    %undervisning och utveckling av forskarutbildningskurser
    % ledning av doktorand- och forskarseminarier
    %tvärvetenskapligt samarbete
    %övrigt
    \subsection{Courses \& Thesis supervision, postgraduate level}

    \subsubsection{Ph.D-thesis supervision}
    - 2017 - ongoing. Co-supervisor for Ph.D.-candidate Jörgen Vikström, Theory of Science, University of Gothenburg.\\
    - 2016 - 2017. Co-supervisor for Ph.D.-candidate Erik Joelsson, Theory of Science, University of Gothenburg. \href{https://gupea.ub.gu.se/handle/2077/51493}{Thesis completed in February 2017}.

    \subsubsection{Teaching, post-graduate level}
    - (2012), Guest lecture, ``Net politics, process philosophy and Deleuze'', Theories of Practical Knowing, HDK/Högskolan för Design och Konsthantverk, Göteborg.\\
    - (2012), Guest lecture and seminar, ``Tyst kunskap'', 2012 års Marstrandsseminarium, Teori \& metod Tystnad, tradition eller ``anything goes''?, Varberg.\\
    - (2016), Guest lecture ``Citizen Science and Biodiversity Research'', BioEnv higher education, Department of Biological and Environmental Sciences, University of Gothenburg.\\
      % Bihandledare Erik joelsson
      % Opponent Eriks slutseminarium
      % Gästföreläsningar  Gästföreläsning Lenas doktorandseminarium,



        %3. Pedagogiskt utvecklingsarbete
        % - utvärderingar
        % - projekt
        % - pedagogiska konferenser
        % - övrigt

        %4. Läromedel och pedagogiska skrifter
        % - tryckt material
        % - digitalt material
        % - övrigt

        %5. Andra uppgifter i anslutning till undervisning på fakultets eller institutionsnivå
        \subsection{Other}
          \begin{itemize}
            % Lägg till när jag fått Lenas intyg att jag gjort kurusplaner etc.
            \item \href{http://files.christopherkullenberg.se/studierektoronline.pdf}{Study Administrator (Studierektor), Spring semester 2014}. % Lena intyg också
            \item \href{http://files.christopherkullenberg.se/IntygREGU.pdf}{2012 - 2014. Member of Rådet för Europastudier vid Göteborgs universitet (REGU)}. %Referens Urban Strandberg
          \end{itemize}

        %6. Övrig pedagogisk meritering
        %T.ex. gästlärare vid utländskt lärosäte.
        %Som underlag för bedömning av den pedagogiska skickligheten bifogas referenser,
        %tjänstgöringsintyg och andra typer av skriftliga värdeomdömen, liksom sammanställ-
        %ningar och uppföljningar av kursvärderingar.





%  _____                                   .___.__
%  /  _  \ ______ ______   ____   ____    __| _/|__|__  ___
% /  /_\  \\____ \\____ \_/ __ \ /    \  / __ | |  \  \/  /
%/    |    \  |_> >  |_> >  ___/|   |  \/ /_/ | |  |>    <
%\____|__  /   __/|   __/ \___  >___|  /\____ | |__/__/\_ \
%       \/|__|   |__|        \/     \/      \/          \/
% All of of this is hyperlinked, so it is uncommented by default for online version.
% Note: these documents contain personnummer.

%\clearpage
%\section{Appendix}

    % Make index, make sure numbers correspond with the overlay belows
%    \begin{enumerate}
%      \item Letter from the President of the European Research Council, Jean-Pierre Bourguignon. 1 page.
%      \item Proofs of course completion, pedagogics in higher education, 2 pages.
%      \item Previous positions, University of Gothenburg, 1 page.
%    \end{enumerate}

%    \clearpage
        % Use the \node at (15, .1) to move around the text saying ``appendix''
%        \includepdf[pagecommand={\begin{tikzpicture}[remember picture, overlay]\node at (15, .1) {Appendix 1};\end{tikzpicture}},pages={1-},scale=.75,lastpage=7]{files/doctoraldegree.pdf}
%        \includepdf[pagecommand={\begin{tikzpicture}[remember picture, overlay]\node at (15, .1) {Appendix 2};\end{tikzpicture}},pages={1-},scale=.75,lastpage=7]{files/examengrundutbildning.pdf}
%        \includepdf[pagecommand={\begin{tikzpicture}[remember picture, overlay]\node at (15, .1) {Appendix 3};\end{tikzpicture}},pages={1-},scale=.75,lastpage=7]{files/intygpedagogisktarbeteGU.pdf}
%        \includepdf[pagecommand={\begin{tikzpicture}[remember picture, overlay]\node at (15, .1) {Appendix 3};\end{tikzpicture}},pages={1-},scale=.75,lastpage=7]{/files/IntygREGU.pdf}
%        \includepdf[pagecommand={\begin{tikzpicture}[remember picture, overlay]\node at (15, .1) {Appendix 5};\end{tikzpicture}},pages={1-},scale=.75,lastpage=7]{files/universitywest.pdf}
%        \includepdf[pagecommand={\begin{tikzpicture}[remember picture, overlay]\node at (15, .1) {Appendix 6};\end{tikzpicture}},pages={1-},scale=.75,lastpage=7]{files/erc.pdf}
%        \includepdf[pagecommand={\begin{tikzpicture}[remember picture, overlay]\node at (15, .1) {Appendix 7};\end{tikzpicture}},pages={1-},scale=1,lastpage=7]{files/hogskolepedagogik.pdf}
%        \includepdf[pagecommand={\begin{tikzpicture}[remember picture, overlay]\node at (15, .1) {Appendix 5};\end{tikzpicture}},pages={1-},scale=1,lastpage=7]{files/anstallningarGU.pdf}
%        \includepdf[pagecommand={\begin{tikzpicture}[remember picture, overlay]\node at (15, .1) {Appendix 7};\end{tikzpicture}},pages={1-},scale=1,lastpage=7]{files/ekvivaleringpedagogik.pdf}
%        \includepdf[pagecommand={\begin{tikzpicture}[remember picture, overlay]\node at (15, .1) {Appendix 8};\end{tikzpicture}},pages={1-},scale=1,lastpage=7]{files/NWOreview.pdf}
%        \includepdf[pagecommand={\begin{tikzpicture}[remember picture, overlay]\node at (10, 2) {Appendix 7 (PER 2016/240)};\end{tikzpicture}},pages={1-},scale=1,lastpage=7]{/files/appliedITevaluation/kullenbergevaluation1.pdf}
%        \includepdf[pagecommand={\begin{tikzpicture}[remember picture, overlay]\node at (10, 2) {Appendix 7 (PER 2016/240)};\end{tikzpicture}},pages={1-},scale=1,lastpage=7]{/files/appliedITevaluation/kullenbergevaluation2.pdf}



\end{document}
