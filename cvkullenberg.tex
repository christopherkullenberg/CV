% Options for packages loaded elsewhere
\PassOptionsToPackage{unicode}{hyperref}
\PassOptionsToPackage{hyphens}{url}
%
\documentclass[
]{article}
\usepackage{lmodern}
\usepackage{amssymb,amsmath}
\usepackage{ifxetex,ifluatex}
\ifnum 0\ifxetex 1\fi\ifluatex 1\fi=0 % if pdftex
  \usepackage[T1]{fontenc}
  \usepackage[utf8]{inputenc}
  \usepackage{textcomp} % provide euro and other symbols
\else % if luatex or xetex
  \usepackage{unicode-math}
  \defaultfontfeatures{Scale=MatchLowercase}
  \defaultfontfeatures[\rmfamily]{Ligatures=TeX,Scale=1}
\fi
% Use upquote if available, for straight quotes in verbatim environments
\IfFileExists{upquote.sty}{\usepackage{upquote}}{}
\IfFileExists{microtype.sty}{% use microtype if available
  \usepackage[]{microtype}
  \UseMicrotypeSet[protrusion]{basicmath} % disable protrusion for tt fonts
}{}
\makeatletter
\@ifundefined{KOMAClassName}{% if non-KOMA class
  \IfFileExists{parskip.sty}{%
    \usepackage{parskip}
  }{% else
    \setlength{\parindent}{0pt}
    \setlength{\parskip}{6pt plus 2pt minus 1pt}}
}{% if KOMA class
  \KOMAoptions{parskip=half}}
\makeatother
\usepackage{xcolor}
\IfFileExists{xurl.sty}{\usepackage{xurl}}{} % add URL line breaks if available
\IfFileExists{bookmark.sty}{\usepackage{bookmark}}{\usepackage{hyperref}}
\hypersetup{
  pdftitle={CV},
  pdfauthor={Christopher Kullenberg},
  hidelinks,
  pdfcreator={LaTeX via pandoc}}
\urlstyle{same} % disable monospaced font for URLs
\setlength{\emergencystretch}{3em} % prevent overfull lines
\providecommand{\tightlist}{%
  \setlength{\itemsep}{0pt}\setlength{\parskip}{0pt}}
\setcounter{secnumdepth}{-\maxdimen} % remove section numbering

\title{CV}
\author{Christopher Kullenberg}
\date{\today}

\begin{document}
\maketitle

\newpage

\hypertarget{formal-education}{%
\subsection{Formal education}\label{formal-education}}

\hypertarget{academic-first-degrees}{%
\subsubsection{Academic first degrees}\label{academic-first-degrees}}

\begin{itemize}
\tightlist
\item
  \href{http://files.christopherkullenberg.se/examengrundutbildningonline.pdf}{B.A.,
  Media \& Communication Science, 2004}.
\item
  \href{http://files.christopherkullenberg.se/examengrundutbildningonline.pdf}{M.A.,
  Theory of Science, 2005}.
\end{itemize}

\hypertarget{doctoral-degree}{%
\subsubsection{Doctoral degree}\label{doctoral-degree}}

\begin{itemize}
\tightlist
\item
  \href{http://files.christopherkullenberg.se/doctoraldegreeonline.pdf}{Ph.D.,
  Theory of Science, 2012.}
\end{itemize}

\hypertarget{promotion-to-associate-professor-docent}{%
\subsubsection{Promotion to Associate Professor
(Docent)}\label{promotion-to-associate-professor-docent}}

\begin{itemize}
\tightlist
\item
  \href{http://files.christopherkullenberg.se/Docentbevis_ChristopherKullenberg_P2017_366_anon.pdf}{Associate
  Professor in Theory of Science (Docent i Vetenskapsteori), 2018-05-09}
\end{itemize}

\hypertarget{other-courses}{%
\subsubsection{Other courses}\label{other-courses}}

\begin{itemize}
\tightlist
\item
  \href{http://files.christopherkullenberg.se/hogskolepedagogikonline.pdf}{2006
  Teaching and Learning in Higher Education}
  (\href{http://files.christopherkullenberg.se/PE0940.pdf}{Högskolepedagogik
  för lärare, PE0940}), 7.5 credits.
\item
  \href{http://files.christopherkullenberg.se/hogskolepedagogikonline.pdf}{2006
  Supervision in Postgraduate Education}
  (\href{http://files.christopherkullenberg.se/PD0180.pdf}{Högskolepedagogisk
  handledning, PD0180}), 7.5 credits.
\item
  \href{http://kursplaner.gu.se/svenska/HPE102.pdf}{2016 Formal credits
  for the course HPE102 Teaching and learning in Higher Education 2:
  Discipline Specific Pedagogic}.
\item
  \href{https://courses.edx.org/certificates/f9c30b3913be4004b95813db59432509}{2016
  MITx 6.00.1x Introduction to Computer Science}.
\item
  2018 LADOK3 Authorization course, University of Gothenburg.
\item
  2018 HPE103 Teaching and Learning in Higher Education 3: Applied
  Analysis, 5 credits (Behörighetsgivande högskolepedagogik 3:
  Självständigt arbete).
\end{itemize}

\hypertarget{employments}{%
\subsection{Employments}\label{employments}}

\hypertarget{current-employment}{%
\subsubsection{Current Employment}\label{current-employment}}

\begin{itemize}
\tightlist
\item
  20180201 - Senior Lecturer in Theory of Science, Dept. of Philosophy,
  Linguistics and Theory of Science, University of Gothenburg. Full
  time. (Application number PAR 2017/92).
\end{itemize}

\hypertarget{research-projects-externally-funded.}{%
\subsubsection{\texorpdfstring{Research projects, externally funded.
}{Research projects, externally funded. }}\label{research-projects-externally-funded.}}

\begin{enumerate}
\def\labelenumi{\arabic{enumi}.}
\tightlist
\item
  Responsible for work package ``Portal construction'', 40\%, 2017-2021,
  ``Arenor för relationsbyggande samverkan genom medborgarforskning'',
  Vinnova, PI: \href{mailto:dick.kasperowski@gu.se}{Dick Kasperowski},
  Grant number: 2017-03527.
\item
  Researcher 50\%, 2015-2018, ``Taking Science to the Crowd:
  Researchers, Programmers and Volunteer Contributors Transforming
  Science Online.'' Marianne \& Marcus Wallenberg Foundation. PI:
  \href{mailto:dick.kasperowski@gu.se}{Dick Kasperowski}, Grant number:
  \href{https://www.wallenberg.com/MMW/projektanslag-2013}{MMW
  2013.0020}.
\item
  Researcher 50\%, 2015-2017, ``The Co-production of Social Science and
  Society: The Case of Happiness studies.'' Swedish Research Council
  (Vetenskapsrådet) PI: Margareta Hallberg, Grant number:
  \href{http://vrproj.vr.se/detail.asp?arendeid=90421}{2012-1117}.
\item
  Researcher 50\%, 2015, ``Subcultures on the Net: Resistance and
  Engagement in Knowledge Practices'', LETStudio, pilot project.
\end{enumerate}

\hypertarget{previous-employments}{%
\subsubsection{Previous employments}\label{previous-employments}}

\begin{itemize}
\tightlist
\item
  \href{http://files.christopherkullenberg.se/universitywest.pdf}{Instructor,
  part time, 2004-2006}. University West.
\item
  \href{http://files.christopherkullenberg.se/anstallningarGU.pdf}{Adjunct
  Lecturer, part time, 2006-2013}. University of Gothenburg.
\item
  \href{http://files.christopherkullenberg.se/anstallningarGU.pdf}{Instructor,
  part time, 2014-2015}. University of Gothenburg.
\item
  \href{http://files.christopherkullenberg.se/anstallningsbeslutboras.pdf}{Senior
  lecturer, part time, 201708-201802}. University of Borås.
\end{itemize}

\hypertarget{administrative-duties}{%
\subsubsection{Administrative duties}\label{administrative-duties}}

\begin{itemize}
\tightlist
\item
  2022 (Aug) - 2023 (Mar). Acting Head of doctoral studies at the
  Department of Philosophy, Linguistics and Theory of Science.
\item
  2018 - 2020. Programme Co-ordinator for the Master's Programme for
  communication officers in the public sector, 60 - 120 credits,
  University of Gothenburg.
\item
  \href{http://files.christopherkullenberg.se/studierektoronline.pdf}{Study
  Administrator, spring semester 2014}. University of Gothenburg.
\end{itemize}

\hypertarget{publications}{%
\subsection{Publications}\label{publications}}

See also \url{https://orcid.org/0000-0002-1577-3570}

\hypertarget{monographs}{%
\subsubsection{Monographs}\label{monographs}}

\begin{enumerate}
\def\labelenumi{\arabic{enumi}.}
\tightlist
\item
  Kullenberg, C. (2012) \emph{The Quantification of Society. A Study of
  a Swedish Research Institute and Survey-Based Social Science},
  Department of Philosophy, Linguistics and Theory of Science,
  University of Gothenburg, Doctoral dissertation,
  \href{https://gupea.ub.gu.se/handle/2077/28807}{ISBN
  978-91-628-8458-1}. 213 pages.
\item
  Kullenberg, C. (2010) \emph{Det Nätpolitiska Manifestet}, Stockholm:
  Ink bokförlag, ISBN 9789197846912. 80 pages.
\end{enumerate}

\hypertarget{journal-articles}{%
\subsubsection{Journal articles}\label{journal-articles}}

\hypertarget{peer-reviewed}{%
\paragraph{Peer reviewed}\label{peer-reviewed}}

\begin{enumerate}
\def\labelenumi{\arabic{enumi}.}
\tightlist
\item
  Frauke Rohden, \textbf{Christopher Kullenberg}, Niclas Hagen, Dick
  Kasperowski (2019) ``Tagging, Pinging and Linking -- User Roles in
  Virtual Citizen Science Forums'', \emph{Citizen science: theory and
  practice} 4(1), p.19. DOI:
  \href{http://doi.org/10.5334/cstp.181}{10.5334/cstp.181}.
\item
  Hallberg, M. \& \textbf{Kullenberg, C.} (2019) ``Happiness Studies:
  Co-Production of Social Science and Social Order'', \emph{Nordic
  Journal of Science and Technology Studies} 7(9)
  \url{https://doi.org/10.5324/njsts.v7i1.2530}.
\item
  Fleischer, Rasmus \& \textbf{Kullenberg, Christopher} (2018)
  ``\href{http://www.cultureunbound.ep.liu.se/article.asp?DOI=10.3384/cu.2000.1525.20180918}{The
  Political Significance of Spotify in Sweden--- Analysing the
  \#backaspotify Campaign using Twitter Data}'', \emph{Culture Unbound},
  Vol 10, Issue 2, DOI 10.3384/cu.2000.1525.20180918.
\item
  \textbf{Kullenberg, Christopher}, Frauke Rohden, Anders Björkvall,
  Fredrik Brounéus, Anders Avellan-Hultman, Johan Järlehed, Sara Van
  Meerbergen, et al.~``What Are Analog Bulletin Boards Used for Today?
  Analysing Media Uses, Intermediality and Technology Affordances in
  Swedish Bulletin Board Messages Using a Citizen Science Approach.''
  \emph{PLOS ONE} 13, no. 8 (August 2018): e0202077.
  \href{https://doi.org/10.1371/journal.pone.0202077}{10.1371/journal.pone.0202077}.
\item
  Marisa Ponti, Thomas Hillman, \textbf{Christopher Kullenberg}, Dick
  Kasperowski (2018) ``Getting it Right or Being Top Rank: Games in
  Citizen Science'', \emph{Citizen Science: Theory and Practice}, 3(1),
  DOI: \href{http://doi.org/10.5334/cstp.101}{10.5334/cstp.101}.
\item
  \textbf{Kullenberg, C.} , \& Nelhans, Gustaf (2017)
  \href{https://dx.doi.org/10.3384/VS.2001-5992.17517}{``Measuring
  Welfare beyond GDP - Objective and Subjective Indicators in Sweden,
  1968-2015''}, \emph{Valuation Studies, 5(1)},
  \href{https://dx.doi.org/10.3384/VS.2001-5992.17517}{10.3384/VS.2001-5992.17517}.
\item
  \textbf{Kullenberg, C.} , \& Dick Kasperowski (2016) ``What is Citizen
  Science? -- A Scientometric Meta-Analysis'', \emph{PLoS ONE}, 11(1):
  e0147152,
  \href{http://dx.doi.org/10.1371/journal.pone.0147152}{10.1371/journal.pone.0147152}.
\item
  \textbf{Kullenberg, C.} (2015) ``Citizen Science as Resistance:
  Crossing the Boundary Between Reference and Representation'',
  \href{https://gup.ub.gu.se/publication/218601-citizen-science-as-resistance-crossing-the-boundary-between-reference-and-representation}{Journal
  of Resistance Studies}, 1(1).
\item
  \textbf{Kullenberg, C.} , \& Nelhans, Gustaf (2015) ``The Happiness
  Turn? Mapping the Emergence of Happiness Studies using Cited
  References'', \emph{Scientometrics}, Volume 103, Issue 2, Page
  615-630,
  \href{http://dx.doi.org/10.1007/s11192-015-1536-3}{10.1007/s11192-015-1536-3}.
\end{enumerate}

\hypertarget{editorship-of-peer-reviewed-journal-issues}{%
\paragraph{Editorship of peer reviewed journal
issues}\label{editorship-of-peer-reviewed-journal-issues}}

\begin{enumerate}
\def\labelenumi{\arabic{enumi}.}
\tightlist
\item
  Dick Kasperowski \& \textbf{Christopher Kullenberg} (2019) ``Special
  Issue: Many Modes of Citizen Science'', \emph{Science and Technology
  Studies}, Vol 32/2019, Issue 2.
\end{enumerate}

\hypertarget{non-peer-review}{%
\paragraph{Non-peer review}\label{non-peer-review}}

\begin{enumerate}
\def\labelenumi{\arabic{enumi}.}
\tightlist
\item
  Kullenberg, C. (2016)
  \href{https://humanit.hb.se/article/view/504/572}{``Sökandet efter den
  digitala politiken''}, \emph{Human IT}, 13(2): 34-46 (book review).
\item
  Kullenberg, C. (2015) ``Internetövervakning som metod'',
  \href{http://gup.ub.gu.se/records/fulltext/220443/220443.pdf}{Ikaros},
  2015(1), pp 9--11.
\item
  Kullenberg, C. (2011) ``Bortom Wikileaks 1. Om nätläckans
  infrastruktur'', \emph{Ord \& Bild}, 2011(1).
\item
  Kullenberg, C. (2009)
  \href{http://www.fiff.de/publikationen/fiff-kommunikation/fk-2009/fiff-ko-1-2009/fiko_1_2009_kullenberg.pdf}{``The
  Social Impact of IT -- Surveillance and Resistance in Present Day
  Conficts''}, \emph{FiFF-Kommunikation}, 09(1).
\item
  Kullenberg, C. , \& Palmås, Karl (2009) ``Contagiontology'',
  \href{http://www.eurozine.com/articles/2009-03-09-kullenberg-en.html}{Eurozine}.
  First published in \emph{Glänta} 2008(4) as ``Smittontologi''.
\item
  Kullenberg, C. (2008) ``Robotar och krig'', \emph{Ord \& Bild},
  2008(5).
\end{enumerate}

\hypertarget{book-chapters}{%
\subsubsection{Book chapters}\label{book-chapters}}

\hypertarget{peer-reviewed-1}{%
\paragraph{Peer reviewed}\label{peer-reviewed-1}}

\begin{enumerate}
\def\labelenumi{\arabic{enumi}.}
\tightlist
\item
  Nielsen, H. L., Rosendal, T., Järlehed, J., \& \textbf{Kullenberg, C.}
  (2020). ``Investigating Bulletin Boards with Students: What Can
  Citizen Science Offer Education and Research in the Linguistic
  Landscape?'' In D. Malinowski, H. H. Maxim, \& S. Dubreil (Eds.),
  \emph{Language Teaching in the Linguistic Landscape: Mobilizing
  Pedagogy in Public Space} (pp.~349--371). Springer International
  Publishing. \url{https://doi.org/10.1007/978-3-030-55761-4_15}
\item
  Kasperowski, Dick; \textbf{Kullenberg, Christopher}; Rohden, Frauke
  (2020) ``The Participatory Epistemic Cultures of Citizen Humanities:
  Bildung and epistemic subjects'', in Hetland et al.~\emph{A History of
  Participation in Museums and Archives}, Routledge: London and New
  York.
\item
  Kasperowski, Dick; \textbf{Kullenberg, Christopher}; Rohden, Frauke
  (2020) ``The epistemology of mobilising citizens in the sciences:
  Tensions in epistemic cultures of contribution and ideals of
  science'', in Mäkitalo et al.~\emph{Designs for Experimentation and
  Inquiry"}, Routledge: London and New York.
\item
  Hård af Segerstad, Y. ; Howes, C. ; Kasperowski, D. \&
  \textbf{Kullenberg, C.} (2017) ``Studying closed communities on-line:
  digital methods and ethical considerations beyond informed consent and
  pseudonymity''. Zimmer, M. \& Kinder-Kurlanda, K. (Eds.).
  \emph{Internet Research Ethics for the Social Age: New Cases and
  Challenges}. Peter Lang.
\item
  Kullenberg, C. (2011) ``Sociology in the Making: Statistics as a
  Mediator between the Social sciences, Practice and the State'', in Ann
  Rudinow Saetnan, Heidi Mork Lomell, Svein Hammer, \emph{The Mutual
  Construction of Statistics and Society}, New York: Routledge.
\end{enumerate}

\hypertarget{editorial-review}{%
\paragraph{Editorial review}\label{editorial-review}}

\begin{enumerate}
\def\labelenumi{\arabic{enumi}.}
\tightlist
\item
  Kullenberg, C. (2014) ``Verkon tunnelit ja niiden kaivajat'', in
  Brunila Mikael \& Kallio Kimmo (eds.), \emph{Verkko suljettu --
  Internet ja avoimuuden rajat}, Helsinki: Into kustannus, ISBN
  978-952-264-248-6.
\item
  Kullenberg, C. (2013) ``Resistance in hybrid networks: The case of
  Telecomix'', in Bueti, Federica (ed.), \emph{Move\ldots{} ment},
  London: Book Works.
\item
  Kullenberg, C. (2009) ``Den panspektriska tidsålderns
  motståndsstrategier'', in Lilja, Mona \& Vinthagen Stellan (eds)
  \emph{Motstånd}, Malmö: Liber AB.
\item
  Kullenberg, C. (2008) ``Tre frågor om sociologins objektivitet och
  förhållande till naturvetenskaperna'', in Hallberg, Margareta (eds),
  \emph{Vi vet något -- Festskrift till Jan Bärmark}, Göteborg:
  Institutionen för idéhistoria och vetenskapsteori, ISBN
  978--91--976239--1--9.
\end{enumerate}

\hypertarget{pre-prints}{%
\paragraph{Pre-prints}\label{pre-prints}}

\begin{enumerate}
\def\labelenumi{\arabic{enumi}.}
\tightlist
\item
  Kullenberg, C. Ponti, M., T Hillman, D Kasperowski, C Kullenberg
  (2017) ``Getting it Right or Being Top Rank: Games in Citizen
  Science'', Socarxiv, \url{https://osf.io/preprints/socarxiv/3qrnc/}
\item
  Kullenberg, C. Kasperowski, D., Kullenberg, C., Mäkitalo, Å. (2017)
  ``Embedding Citizen Science in Research: Forms of engagement,
  scientific output and values for science, policy and society'',
  Socarxiv, \url{https://osf.io/preprints/socarxiv/tfsgh/}
\end{enumerate}

\hypertarget{other-publications}{%
\subsubsection{Other publications}\label{other-publications}}

\hypertarget{reports}{%
\paragraph{Reports}\label{reports}}

\begin{enumerate}
\def\labelenumi{\arabic{enumi}.}
\tightlist
\item
  Kasperowski, Dick \& Kullenberg, Christopher (2018)
  ``\href{http://www.formas.se/sv/Om-Formas/Formas-Publikationer/Rapporter/Medborgarforskning-och-vetenskapens-demokratisering/}{Medborgarforskning
  och vetenskapens demokratisering}'', Forskningsrådet Formas, Rapport
  R3:2018, ISBN 978-91-540-6102-0.
\item
  Kullenberg, C. Ulrike Sturm, Sven Schade, Luigi Ceccaroni, Margaret
  Gold, Christopher C. M. Kyba, Bernat Claramunt, Muki Haklay, Dick
  Kasperowski, Alexandra Albert, Jaume Piera, Jonathan Brier,
  Christopher Kullenberg, Soledad Luna (2017)
  \href{https://doi.org/10.3897/rio.4.e23394}{``Defining principles for
  mobile apps and platforms development in citizen science''}, Workshop
  Report, Research Ideas and Outcomes 4: e23394 (04 Jan 2018).
\item
  Kullenberg, C. Björkvall, Anders, Johan Järlehed, Christopher
  Kullenberg, Helle Lykke Nielsen, Andreas Nord, Tove Rosendal, Sara Van
  Meerbergen \& Gustav Westberg (2017)
  \href{https://www.forskarfredag.se/filer/ff2016-anslagstavlan-slutrapport.pdf}{"Slutrapport
  Anslagstavlan - Forskarfredags Massexperiment 2016}, VA-rapport
  2017:1, Red. Fredrik Bronéus, Stockholm: Vetenskap och Allmänhet, ISSN
  1653-6843.
\item
  Kullenberg, C. Kullenberg, Christopher (2017)
  \href{http://www.slu.se/globalassets/ew/subw/lifewatch/publikationer/slw-summary-report-web-170622.pdf}{Open
  data - buzz word or virtual opportunities?}, in \emph{Swedish
  LifeWatch -- a national e-infrastructure for biodiversity data},
  ArtDatabanken SLU, ISBN 978-91-87853-17-3.
\end{enumerate}

\hypertarget{preface-and-review-of-translation}{%
\paragraph{Preface and review of
translation}\label{preface-and-review-of-translation}}

\begin{enumerate}
\def\labelenumi{\arabic{enumi}.}
\tightlist
\item
  Kullenberg, C. (2012) ``Introduktion'', in \emph{Tre Klassiska Texter
  -- Émile Durkheim, Gabriel Tarde, Max Weber}, fackgranskad av
  Christopher Kullenberg, Göteborg: Korpen Koloni.
\end{enumerate}

\hypertarget{proceedings-peer-reviewed}{%
\paragraph{Proceedings: Peer reviewed}\label{proceedings-peer-reviewed}}

\begin{enumerate}
\def\labelenumi{\arabic{enumi}.}
\tightlist
\item
  Nord, Andreas, Järlehed, Johan \& Kullenberg, Christopher (2019) ``För
  anslagstavlan i tiden. Vad händer på svenska anslagstavlor?'', in
  Bianchi, Marco, Håkansson, David, Melander, Björn, Pfister, Linda,
  Westman, Maria \& Östman, Carin (eds.), Svenskans beskrivning 36.
  Förhandlingar vid trettiosjätte sammankomsten. Uppsala 25--27 oktober
  2017. Uppsala: Institutionen för nordiska språk vid Uppsala
  universitet. S. 179--194.
\end{enumerate}

\hypertarget{conference-contributions-peer-reviewed}{%
\paragraph{Conference contributions, peer
reviewed}\label{conference-contributions-peer-reviewed}}

\begin{enumerate}
\def\labelenumi{\arabic{enumi}.}
\tightlist
\item
  Kasperowski, Dick., Kullenberg, Christopher., Rohden, Frauke. (2018)
  \href{https://nomadit.co.uk/easst/easst2018/conferencesuite.php/panels/Views/All\%20Panels}{``Epistemic
  cultures in citizen science and humanities: Distribution, epistemic
  subjects, programs and anti-programs''}, \emph{EASST2018: Meetings -
  Making Science, Technology and Society together}, Lancaster
  University, UK.
\item
  Andreas Nord, Johan Järlehed \& Christopher Kullenberg,
  \href{http://www2.nordiska.uu.se/konferens/svebe36/sammandrag/nord-jarlehed-kullenberg/}{``För
  anslagstavlan i tiden: Vad händer på svenska anslagstavlor?''},
  \emph{Svenskans beskrivning}, Uppsala University, 25--27 october 2017.
\item
  Kullenberg, C. (2017) ``Talking about \#citizenscience'',
  \href{http://socav.gu.se/english/research/third-nordic-science-and-technology-studies-conference}{Third
  Nordic Science and Technology Conference, Göteborg, May 31-June2}.
\item
  Kasperowski, D. ; Kullenberg, C. (2016) ``Citizen Humanities:
  Configuring Interpretation and Perception for Participation''.
  \emph{Citizen Science -- Innovation in Open Science, Society and
  Policy}, 19--21 May 2016, Berlin, 26-27.
\item
  Nelhans, G. ; Kullenberg, C. (2016) ``Happiness as a Valuation of
  Nations: From Margin to Indicator''. \emph{4S/EASST Conference},
  Barcelona 2016.
\item
  Ponti, M; Hagen, N; Hillman, T; Kasperowski, D; Kullenberg, C;
  Stankovic, I (2015) ``Designing Futures for Learning in the Crowd: New
  Challenges and Opportunities for CSCL''. In: Lindwall, O., Häkkinen,
  P., Koschman, T. Tchounikine, P. \& Ludvigsen, S. (Eds.) (2015).
  \emph{Exploring the Material Conditions of Learning: The Computer
  Supported Collaborative Learning (CSCL) Conference}, Vol. 2 s.
  885-888.
\item
  Kasperowski, D; Hagen, N; Kullenberg, C; Walford, A; Smith, K, V;
  Liborion, M; Prutzer N \& Hamid, S, T (2015) ``Joining Reference and
  Representation --- Citizen Science as Resistance Practice''.
  \emph{Society for Social Studies of Science 2015 Annual Meeting},
  November 11-14 Denver, Colorado.
\item
  Kullenberg, C. (2008) ``The Panspectrocist Abstract Machine 1.
  Machinic Phyla, Territorialities and Social Diagrams''. \emph{The
  First International Deleuze Studies Conference}, 1 Monday: 11/08/08.
\item
  Kullenberg, C. (2008) ``The Objects and Objections of the Social
  Sciences -- The Re-discovery of Epistemic Practice as a Tool for
  Enhancing Public Participation''. \emph{Ironists, Reformers or Rebels?
  The Role of the Social Sciences in Participatory Policy Making}, June
  26-27th 2008 at Collegium Helveticum, UZH/ETH Zürich, Switzerland.
\item
  Kullenberg, C. (2008) ``From State Sociology to Centres of
  Calculation: the Swedish case''. \emph{Acting with Science, Technology
  and Medicine 4S-EASST 2008 Rotterdam}, the Netherlands, August 20-23
  2008. 367-368.
\end{enumerate}

\hypertarget{research-notes}{%
\paragraph{Research notes}\label{research-notes}}

\begin{enumerate}
\def\labelenumi{\arabic{enumi}.}
\tightlist
\item
  Kullenberg, C. (2017) ``Bibliometric probing of the concept `open
  science' - a notebook'',
  \url{https://github.com/christopherkullenberg/openscienceliterature/}.
\end{enumerate}

\hypertarget{peer-reviewed-encyclopedia-entries-and-bibliographies}{%
\paragraph{Peer reviewed encyclopedia entries and
bibliographies}\label{peer-reviewed-encyclopedia-entries-and-bibliographies}}

\begin{enumerate}
\def\labelenumi{\arabic{enumi}.}
\tightlist
\item
  Kullenberg, C. (2021), ``Citizen Science'', In Environmental Science.
  Oxford University Press.
  https://doi.org/10.1093/obo/9780199363445-0133
\item
  Kullenberg, C. (2023), ``Citizen Science'', In: Encyclopedia of Social
  innovation, edit. by J. Howaldt and C. Kaletka, Cheltenham: Edward
  Elgar.
\end{enumerate}

\hypertarget{popularizations}{%
\paragraph{Popularizations}\label{popularizations}}

\begin{enumerate}
\def\labelenumi{\arabic{enumi}.}
\tightlist
\item
  Kullenberg, C. (2022) Guest blogging, \emph{Curie},
  \url{https://www.tidningencurie.se/kronikorer/christopher-kullenberg}
\item
  Kullenberg, C. (2015)
  \href{http://www.svd.se/tankesprang-som-kopplar-ihop-varlden}{``Tankesprång
  som kopplar ihop världen''}, \emph{SvD Under Strecket}, 4 September,
  2015.
\item
  Kullenberg, C. (2014)
  \href{http://www.svd.se/kultur/understrecket/gazakonfliktens-underjordiska-liv_3776810.svd}{``Gazakonfliktens
  underjordiska liv''}, \emph{SvD Under Strecket}, 26 Juli, 2014.
\item
  Kullenberg, C. (2011)
  \href{http://www.fria.nu/artikel/88431}{``Internet nära oss''},
  \emph{Fria tidningen} 2011-05-30.
\end{enumerate}

\hypertarget{opinion-pieces}{%
\paragraph{Opinion pieces}\label{opinion-pieces}}

\begin{enumerate}
\def\labelenumi{\arabic{enumi}.}
\tightlist
\item
  Kullenberg, C. 20160402, Svenska Dagbladet (with Niclas Hagen),
  \href{http://www.svd.se/empati-och-solidaritet-tar-nya-former-pa-natet}{``Empati
  och solidaritet tar nya former på nätet''}.
\item
  Kullenberg, C. 20160322, Expressen,
  \href{http://www.expressen.se/debatt/finns-ingen-anledning-att-cyberkapprusta/}{``Det
  finns ingen anledning att cyberkapprusta''}.
\item
  Kullenberg, C. 20151119, Expressen,
  \href{http://www.expressen.se/debatt/overvakningsfragor-maste-fa-ta-sin-tid/}{``Övervakningsfrågor
  måste få ta sin tid''}.
\item
  Kullenberg, C. 20130907, Expressen,
  \href{http://www.expressen.se/debatt/nu-maste-vi-stanga-bakdorrarna/}{``Nu
  måste vi stänga bakdörrarna''}.
\item
  Kullenberg, C. 20130611, Expressen,
  \href{http://www.expressen.se/debatt/det-stora-avslojandet-aterstar-for-snowden/}{``Det
  stora avslöjandet återstår för Snowden''}.
\item
  Kullenberg, C. 20110221, Expressen,
  \href{http://www.expressen.se/debatt/christopher-kullenberg-akrobatikrekord-i-regeringskonst/}{``Akrobatirekord
  i regeringskonst''}.
\item
  Kullenberg, C. 20101118, SVT Debatt, \href{}{``Wikileaks är större än
  den häktade Julian Assange''}.
\item
  Kullenberg, C. 20100928, SVT Debatt, \href{}{``Socialdemokraterna bör
  vara stolta över att ha banat väg för The Pirate Bay''}.
\item
  Kullenberg, C. 20100907, Svenska Dagbladet,
  \href{http://www.svd.se/internetoperatorerna-maste-skyddas-fran-upphovsrattsindustrin}{``Internetoperatörerna
  måste skyddas från upphovsrättsindustrin''}.
\item
  Kullenberg, C. 20100717, Svenska Dagbladet (with Marcin de Kaminski),
  \href{http://www.svd.se/internet-ar-redan-trasigt}{``Internet är redan
  trasigt''}.
\item
  Kullenberg, C. 20090709, Newsmill, \href{}{``Google borde stå på
  internetmedborgarnas och inte diktaturernas sida''}.
\item
  Kullenberg, C. 20090617, Newsmill, \href{}{``Så hjälper The Pirate Bay
  motståndet i Iran''}.
\item
  Kullenberg, C. 20090506, Newsmill, \href{}{``Bloggarna står för den
  bästa rapporteringen kring EU:s viktigaste fråga''}.
\item
  Kullenberg, C. 20090504, Newsmill, \href{}{``Telekompaketet en
  medborgarrättslig katastrof''}.
\item
  Kullenberg, C. 20090429, Expressen,
  \href{http://www.expressen.se/debatt/sverige-maste-raddas-undan-var-internetfientliga-regering/}{``Sverige
  måste räddas undan vår internetfientliga regering''}.
\item
  Kullenberg, C. 20081219, Svenska Dagbladet, \href{}{``Nya politiska
  idéer smittar som virus''}.
\item
  Kullenberg, C. 20081113, Sydsvenskan,
  \href{http://www.sydsvenskan.se/2008-11-12/piratjagarlagen}{``Piratjägarlagen''}.
\item
  Kullenberg, C. 20081007, Svenska Dagbladet,
  \href{http://www.svd.se/det-digitala-livet-ar-inte-mindre-verkligt}{``Det
  digitala livet är inte mindre verkligt''}.
\item
  Kullenberg, C. 20080926, Expressen, \href{}{``De förtjänar inte kallas
  liberaler''}.
\item
  Kullenberg, C. 20080916, Expressen (with Mark Klamberg \& Karl
  Palmås),
  \href{http://www.expressen.se/kultur/silence-fiction/}{``Silence
  Fiction''}.
\item
  Kullenberg, C. 20080916, Aftonbladet (with Emina Karic),
  \href{}{``FRA:s signalspaning liknar Sovjetssystemets centraldator''}.
\item
  Kullenberg, C. 20080916, Sydsvenskan, \href{}{``Kreativitet bästa
  motståndet mot FRA''}.
\item
  Kullenberg, C. 20080715, Expressen, (with Evelina Wahlqvist),
  \href{http://www.expressen.se/debatt/ryssland-ar-inte-hotet-bergling/}{``Ryssland
  är inte hotet Bergling!''}.
\item
  Kullenberg, C. 20080709, Svenska Dagbladet,
  \href{http://www.svd.se/fra-kan-visst}{``FRA kan visst!''}.
\item
  Kullenberg, C. 20080704, Aftonbladet, \href{}{``Anti-FRA -- den nya
  liberalismen''}.
\item
  Kullenberg, C. 20080616, Expressen,
  \href{http://www.expressen.se/debatt/fra-lagen-hindrar-fria-forskningen/}{``FRA
  hotar den fria forskningen''}.
\item
  Kullenberg, C. 20080616, Svenska Dagbladet,
  \href{http://www.svd.se/meddelarskyddet-bort-med-ett-knapptryck}{``Meddelarskyddet
  borta med ett knapptryck''}.
\item
  Kullenberg, C. 20061011, Svenska Dagbladet,
  \href{http://www.svd.se/bodstrom-forlorar-snart-natkontrollen}{``Bodström
  förlorar snart nätkontrollen''}.
\item
  Kullenberg, C. 20060806, Svenska Dagbladet,
  \href{http://www.svd.se/vi-har-inget-att-oroa-oss-for}{``Vi har inget
  att oroa oss för''}.
\end{enumerate}

\hypertarget{popularizations-book-chapters-and-articles}{%
\paragraph{Popularizations: book chapters and
articles}\label{popularizations-book-chapters-and-articles}}

\begin{enumerate}
\def\labelenumi{\arabic{enumi}.}
\tightlist
\item
  Kullenberg, C. (2016) ``Alla kan forska - Vad är medborgarvetenskap
  och hur skiljer den sig från vanlig vetenskap'', \emph{Modern
  filosofi}, 2016(3), 68-69.
\item
  Kullenberg, C. (2013) ``Nätets Geopolitik'', in \emph{Det är vår värld
  -- tio unga röster om global solidaritet}, Stockholm: A-smedjan.
\item
  Kullenberg, C. (2010) ``Teknik är samhället gjort hållbart\ldots{}
  eller?'', in Tovhult Klara (eds) \emph{Kunskap, kommunikation,
  kontroll -- Drömmar och farhågor i informationssamhället}, Stockholm:
  Sharing is Caring Förlag.
\end{enumerate}

\hypertarget{collections}{%
\paragraph{Collections}\label{collections}}

\begin{enumerate}
\def\labelenumi{\arabic{enumi}.}
\tightlist
\item
  Kullenberg, C. , \& Lehne, J. (2009) \emph{Resistance Studies Reader
  2008}, London and Gothenburg: Resistance Studies Network.
\end{enumerate}

\hypertarget{open-data-repositories}{%
\paragraph{Open data repositories}\label{open-data-repositories}}

\begin{enumerate}
\def\labelenumi{\arabic{enumi}.}
\tightlist
\item
  Kullenberg, C. (2015) \& Nelhans, Gustaf. (2015). The happiness turn?
  Mapping the Emergence of ``Happiness Studies'' Using Cited References.
  Version 1.0. Svensk Nationell Datatjänst.
  \href{http://dx.doi.org/10.5878/002633}{DOI 10.5878/002633}.
\end{enumerate}

\hypertarget{software}{%
\paragraph{Software}\label{software}}

\begin{enumerate}
\def\labelenumi{\arabic{enumi}.}
\tightlist
\item
  Kullenberg, C (2018) \emph{Flashbackscraper}, scraper software for the
  Flashback.org internet forum,
  \url{https://github.com/christopherkullenberg/flashbackscraper}.
\item
  Kullenberg, C (2018) \emph{\#backaspotify} code repository,
  Visualisation and anonymisation software for Twitter analysis,
  \url{https://github.com/christopherkullenberg/backaspotify}.
\item
  Kullenberg, C (2018) \emph{Talk Analyzer}, forum analysis software,
  \url{https://github.com/christopherkullenberg/talk-analyzer},
  Kullenberg C (2019) talk-analyzer. Figshare. DOI:
  10.6084/m9.figshare.7571003.v1.
\item
  Kullenberg, C. (2015) \emph{Cite.py}, Citation analysis software,
  \url{https://github.com/christopherkullenberg/Citepy}.
\item
  Kullenberg, C. (2016)
  \href{http://offentligautredningar.flov.gu.se/}{offentligautredningar.flov.gu.se},
  Search engine for Statens offentliga utredningar,
  \href{https://github.com/christopherkullenberg/offentligautredningar.se}{github.com/christopherkullenberg/offentligautredningar.se}.
\item
  Kullenberg, C. (2016)
  \emph{\href{https://genuskollen.se}{Genuskollen.se}}, Gender counter
  algorithm based on names,
  \href{https://github.com/christopherkullenberg/gendercounter}{github.com/christopherkullenberg/gendercounter}.
\end{enumerate}

\hypertarget{research-collaborations}{%
\subsection{Research Collaborations}\label{research-collaborations}}

\begin{enumerate}
\def\labelenumi{\arabic{enumi}.}
\tightlist
\item
  Member of the
  \href{https://www.scishops.eu/export-and-advisory-board/\#16}{Expert
  and advisory board for Scishops.eu}.
\item
  Member of the
  \href{https://citizenscienceassociation.org/overview/steering-committees/\#metadata}{Citizen
  Science Association Working Group for Data and Metadata.}
\item
  Member of the
  \href{http://letstudio.gu.se/members/christopher-kullenberg}{Learning
  and Media Technology Studio, LETStudio, University of Gothenburg.}.
\item
  2016-2018. Board Member of the
  \href{http://cassirer.se/sallskapet/styrelse/}{Swedish Ernst Cassirer
  Society}.
\item
  Member of the scientific advisory board of the project
  \href{http://anslag.rj.se/sv/anslag/52553}{``Minnen av Selma
  Lagerlöf'', funded by Riksbankens Jubileumsfond}.
\end{enumerate}

\hypertarget{referee-work}{%
\subsection{Referee work}\label{referee-work}}

\hypertarget{expert-for-employment-or-promotion}{%
\subsubsection{Expert for employment or
promotion}\label{expert-for-employment-or-promotion}}

\begin{enumerate}
\def\labelenumi{\arabic{enumi}.}
\tightlist
\item
  Reviewer for application as Ph.D.~candidate in Theory of Science,
  Dept. of Philosophy, Linguistics and Theory of Science, University of
  Gothenburg, (2017).
\end{enumerate}

\hypertarget{pre-publication-reviews}{%
\subsubsection{Pre-publication reviews}\label{pre-publication-reviews}}

(see also: \url{publons.com/a/1172130/})

\begin{enumerate}
\def\labelenumi{\arabic{enumi}.}
\tightlist
\item
  \emph{New Media and Society} (2018)
\item
  \emph{Citizen Science: Theory and Practice} (2018)
\item
  \emph{Citizen Science: Theory and Practice} (2018)
\item
  \emph{Journal of Science Communication} (2018)
\item
  \emph{Journal of Resistance Studies} (2017)
\item
  \emph{Cities} (2017)
\item
  \emph{Lychnos} (2017)
\item
  \emph{PLOS ONE} (2016)
\item
  \emph{Nordic Journal of Science and Technology Studies} (2016)
\item
  \emph{Journal of the Royal Society of New Zealand} (2015)
\item
  \emph{Higher Education} (2015)
\item
  \emph{Organization Studies} (2015)
\item
  \emph{Journal of Resistance Studies} (2015)
\item
  \emph{Journal of Resistance Studies} (2015)
\end{enumerate}

\hypertarget{research-applications-reviews}{%
\subsubsection{Research applications
reviews}\label{research-applications-reviews}}

\begin{enumerate}
\def\labelenumi{\arabic{enumi}.}
\tightlist
\item
  \href{http://files.christopherkullenberg.se/NWOreview.pdf}{Netherlands
  Organisation for Scientific Research (NWO), Veni Research Proposal}.
  (2017)
\item
  \href{http://files.christopherkullenberg.se/erc.pdf}{ERC Consolidator
  Grant Call 2015, European Research Council}. (2015)
\end{enumerate}

\hypertarget{editorial-board-membership}{%
\subsubsection{Editorial board
membership}\label{editorial-board-membership}}

\begin{enumerate}
\def\labelenumi{\arabic{enumi}.}
\tightlist
\item
  \href{http://resistance-journal.org/the-old-rs-mag/}{2008 1. 2010.
  Editor and founder, Resistance Studies Magazine.}
\item
  \href{http://resistance-journal.org/editorialboard/}{2015 1. present,
  Editorial board member, Resistance Studies Journal.}
\end{enumerate}

\hypertarget{public-outreach-and-extra-academic-collaborations}{%
\subsection{Public outreach and extra-academic
collaborations}\label{public-outreach-and-extra-academic-collaborations}}

\hypertarget{interviews-in-mass-media}{%
\subsubsection{Interviews in mass
media}\label{interviews-in-mass-media}}

\begin{enumerate}
\def\labelenumi{\arabic{enumi}.}
\tightlist
\item
  20210124 - \href{https://sverigesradio.se/avsnitt/1630734}{P3
  Dokumentär ``Arabiska våren''}.
\item
  20200307 - ``Forskare varnar: kan kapa din dator'',
  \href{https://www.svt.se/nyheter/forskaren-varnar-kan-kapa-din-dator}{SVT
  nyheter}.
\item
  20180904 - Göteborgsforskning: Klassisk annonsering fortfarande
  poppis, GöteborgDirekt.
\item
  20180903 - Metoden medborgarforskning innebär att forskare tar hjälp
  av allmänheten, Sveriges Radio P4.
\item
  20180831 - Anslagstavlan används fortfarande och är populär, Sveriges
  Radio P4.
\item
  20180828 - ``Analoga anslagstavlor behövs visst..'', Forskning.se.
\item
  20180828 - ``Anslagstavlor fortfarande poppis'', SVT Nyheter Väst.
\item
  20180412 - Computer Sweden,
  \href{https://computersweden.idg.se/2.2683/1.700745/forskning-data}{``Stort
  skifte väntar när forskningens data öppnas upp''}.
\item
  20180314 - Göteborgs Posten,
  \href{http://www.gp.se/nyheter/ekonomi/experten-d\%C3\%A4rf\%C3\%B6r-skadar-tystnaden-google-1.5393870}{``Expert:
  Tystnaden skadar Google''}.
\item
  20180308 - Dagens Nyheter,
  \href{https://www.dn.se/nyheter/sverige/hatkontots-filmer-tillbaka-pa-youtube/}{``Hatkontots
  filmer tillbaka på Youtube''}.
\item
  20180305 - Dagens Nyheter,
  \href{https://www.dn.se/nyheter/vetenskap/metoo-koden-som-forandrade-varlden/}{``Metoo
  - koden som förändrade världen''}.
\item
  20180131 - SR P3, PP3, \href{http://t.sr.se/2Gv0I2K}{``Porrbotar och
  ord man borde sluta med''}.
\item
  20180123 - Sputnik News,
  \href{https://sputniknews.com/viral/201801231060991240-sweden-bots-porn-storm/}{``Swedish
  Group Rises Against Wave of Porn Bots Flooding Social Networks''}.
\item
  20180123 - Aftonbladet,
  \href{https://www.aftonbladet.se/nyheter/a/On3x0w/svenska-twittrare-invaderade-av-porrbotar}{``Svenska
  twittrare invaderade av porrbotar''}.
\item
  20180122 - SVT Nyheter,
  \href{https://www.svt.se/nyheter/inrikes/de-kartlagger-porrbotarna-pa-twitter}{``De
  kartlägger porrbotarna på Twitter''}.
\item
  20170503 - SVT Nyheter,
  \href{https://www.svt.se/nyheter/lokalt/vast/elever-har-kartlagt-anslagstavlor-i-hela-landet}{``Kartläggning:
  Detta står på anslagstavlor''}.
\item
  20170503 - SR P4 Östergötland,
  \href{http://t.sr.se/2pdhEBb}{``Forskare: Inbjudningar vanligast på
  anslagstavlor''}.
\item
  20170117 - Curie,
  \href{http://www.tidningencurie.se/nyheter/2017/01/17/trycket-okar-gor-forskningsdata-tillgangliga/}{``Trycket
  ökar: Gör forskningsdata tillgängliga''}.
\item
  20160007 - GU-journalen 5-2016,
  \href{https://issuu.com/universityofgothenburg/docs/gu-journalen5-2016/34}{``Anslagstavlan''}.
\item
  20161008 - Dagens Nyheter,
  \href{http://www.dn.se/nyheter/sverige/synen-pa-oppen-publicering-delar-forskarna/}{``Synen
  på öppen publicering delar forskarna''}.
\item
  20161005 - Sveriges Radio, Kulturnytt,
  \href{http://sverigesradio.se/sida/artikel.aspx?programid=478\&artikel=6533649}{``Facebooks
  balansgång mellan globala krav''}.
\item
  20161002 - Arbetarbladet,
  \href{http://www.arbetarbladet.se/gavleborg/sandviken/vilken-betydelse-har-anslagstavlan-i-dag}{"Vilken
  betydelse har anslagstavlan idag?}.
\item
  20160930 - SR P1, \href{http://t.sr.se/2dszL0a}{Vetenskapsradions
  veckomagasin}.
\item
  20160930 - GU-Journalen,
  \href{https://issuu.com/universityofgothenburg/docs/guj4-2016}{``Macchiarini
  did not operate in a vacuum''}.
\item
  20160929 - Dagens Samhälle, ``Elever med mobilapp fångar analoga
  anslag''.
\item
  20160811 - DagensETC, ``Forskningen har flyttat in i hemmet''.
\item
  20160915 - Språktidningen,
  \href{http://spraktidningen.se/blogg/skolelever-far-prova-pa-forskarlivet}{"Skolelever
  får pröva på forskarlivet}.
\item
  20160530 - P4 Östergötland, \href{http://t.sr.se/1NY9nMZ}{``Landets
  anslagstavlor ska kartläggas''}.
\item
  20160523 - Piteåtidningen --
  \href{http://www.pt.se/nyheter/pitea/elever-ska-fota-anslagstavlor-10045575.aspx}{``Elever
  ska fota anslagstavlor''}.
\item
  20160316 - Svenska Dagbladet,
  \href{http://www.svd.se/sanning-inte-viktigt-for-ett-drev/om/natdrev}{``Sanningen
  inte viktig för ett drev''}.
\item
  20160222 - Sveriges Radio 1. P1 Morgon:
  \href{http://t.sr.se/1KBHSak}{``Varför vägrar Apple hacka en telefon
  åt FBI''}.
\item
  20160218 - Götheborske Spionen,
  \href{http://www.spionen.se/140-redaktionellt/reportage/feature/1593-sa-oevervakas-du-via-naetet}{``Så
  övervakas du via nätet''}.
\item
  20160217 - Sesam, ``Vanliga människor hjälper forskare''.
\item
  20160211 - Fria tidningen,
  \href{http://www.fria.nu/artikel/121724}{``Allmänheten är den nya
  forskaren''}.
\item
  20160115 - GU-journalen,
  \href{https://issuu.com/universityofgothenburg/docs/guj1-2016/32}{``Vi
  behöver en lag för internet''}.
\item
  20160118 - SR P1 Vetenskapsradion, \href{http://t.sr.se/239vggG}{Här
  rycker medborgarforskaren in.}
\item
  20160118 - Miljöaktuellt,
  \href{http://miljoaktuellt.se/vanligare-att-forskare-tar-hjalp-av-allmanheten/}{``Vanligt
  med medborgarforskning på miljöområdet''}.
\item
  20160119 - Discover Magazine,
  \href{http://blogs.discovermagazine.com/inkfish/2016/01/19/what-is-citizen-science-good-for-birds-butterflies-big-data/\#.Vp9DvPGFB24}{``What
  Is Citizen Science Good For? Birds, Butterflies, Big Data''}.
\item
  20160120 - Journal of Science Communication,
  \href{http://jcom.sissa.it/node/3076}{``New study on Citizen Science's
  impact on scientific research''}.
\item
  20151207 - Sveriges Radio 1. P1 Morgon,
  \href{http://t.sr.se/1XMYemi}{``Vad innebär de nya åtgärderna mot
  terrorism?''}.
\item
  20150729 - ARTE, Tracks,
  \href{http://tracks.arte.tv/fr/christopher-kullenberg-world-war-web}{``Christopher
  Kullenberg, World War Web''}.
\item
  20150601 - Hallands Nyheter, ``Kommunen betalar med informationen''.
\item
  20150323 - Sveriges Television Vetenskapens Värld,
  \href{https://youtu.be/vG8sZQnU7mU?t=18m1s}{"Medborgarforskning.}
\item
  20150103 - SVT Nyheter,
  \href{http://www.svt.se/nyheter/utrikes/gmail-senaste-malet-for-kinas-natcensur?cmpid=del:pd:ny:20160803:gmail-senaste-malet-for-kinas-natcensur:nyh}{``Gmail
  senaste offret för Kinas nätcensur''}.
\item
  20141108 - Sydsvenskan, ``Anonymitet hotad för knarkköpare''.
\item
  20140329 - Sydsvenskan,
  \href{http://www.sydsvenskan.se/2014-03-29/okad-avlyssning-kraver-nya-motvapen}{``Ökad
  avlyssning kräver nya motvapen''}.
\item
  20131021 - SR P4, \href{http://t.sr.se/NC8sUF}{``Övervakningens
  nödutgångar''}.
\item
  20130912 - Aftonbladet,
  \href{http://www.aftonbladet.se/nyheter/vetmer/article17455373.ab}{``Ditt
  finger i deras händer''}.
\item
  20130907 - Information.dk,
  \href{http://www.information.dk/471440}{``Det er godt, at vi smadrer
  myten om det frie internet''}.
\item
  20130906 - Sveriges Radio P1, \href{http://t.sr.se/OfP0gG}{"Hur trygga
  är vi på nätet?}
\item
  20130719 - Dagens Nyheter,
  \href{http://www.dn.se/kultur-noje/sjalvforstorande-bilder-gor-succe/}{``Självförstörande
  bilder gör succe''}.
\item
  20130615 - Helsingborgs Dagblad,
  \href{http://www.hd.se/2013-06-15/storebror-ser-dig1.\%20Kullenberg,\%20C.\%20men-bryr-vi-oss}{"Storebror
  ser dig, men bryr vi oss?}
\item
  20130613 - Arbetaren, ``Obama-administrationen måste förklara sig''.
\item
  20130607 - Sverige Radio P3,
  \href{http://t.sr.se/MiZb35}{"Korrerapporten.}
\item
  20130601 - Utbildningsradion,
  \href{http://urskola.se/Produkter/176858-UR-Samtiden-Overvakning-och-kontroll-Internetovervakning-och-revolutioner}{``Internetövervakning
  och revolutioner''.}
\item
  20130217 - Sydsvenskan,
  \href{http://www.sydsvenskan.se/2013-02-17/tecken-i-tiden}{``Tecken i
  tiden''}.
\item
  20130214 - Internetworld,
  \href{http://www.idg.se/2.1085/1.492033/han-driver-pa-for-ett-oppet-nat}{``Han
  driver på för ett öppet nät.''}.
\item
  2013(Frühjahr) - Revue, Magaziene for the Next Society, ``On and off
  the grid. Interview: Christopher Kullenberg speaks with Tim Vogler and
  Jan Bathel'', ISBN 978-36-98155008-1-8.
\item
  20121004 - Sveriges Radio Studio ett,
  \href{http://t.sr.se/1foOdCK}{``Simulerade och verkliga
  nätattacker.''}
\item
  20120430 - Sveriges Radio Studio ett,
  \href{http://t.sr.se/1cRSOfk}{``Kan man lita på Wikipedia?''}
\item
  20120321 - Nerikes Allehanda,
  \href{http://na.se/nyheter/orebro/1.1589481--de-som-inte-vill-blir-inte-upptackta-}{``De
  som inte vill bli upptäckta''}.
\item
  20120306 - Internetworld,
  \href{http://www.idg.se/2.1085/1.435933/kreativiteten-ar-inte-individuell}{``Kreativiteten
  är inte individuell''}.
\item
  20111226 - Forbes,
  \href{http://www.forbes.com/sites/andygreenberg/2011/12/26/meet-telecomix-the-hackers-bent-on-exposing-those-who-censor-and-surveil-the-internet/}{``Meet
  Telecomix, The Hackers Bent On Exposing Those Who Censor And Surveil
  The Internet''}.
\item
  20111204 - SR P4 Jönköping, \href{http://t.sr.se/1gy095E}{"Christopher
  från Bodafors är Årets svensk}.
\item
  2011 - ARTE, TRACKS, ``Christopher Kullenberg'', Also on Youtube:
  \url{https://www.youtube.com/watch?v=a3ppeRZHVcY}.
\item
  20111206 - Voxeurop,
  \href{http://www.voxeurop.eu/en/content/article/1254651-cyber-revolutionary-tahrir-square}{``The
  cyber-revolutionary on Tahrir Square''}.
\item
  20111206 - Washington Post,
  \href{https://www.washingtonpost.com/lifestyle/style/the-hacktivists-of-telecomix-lend-a-hand-to-the-arab-spring/2011/12/05/gIQAAosraO_story.html}{``The
  `hacktivists' of Telecomix lend a hand to the Arab Spring''}.
\item
  20111201 - Fokus,
  \href{http://www.fokus.se/2011/12/med-datorn-som-vapen/}{``Med datorn
  som vapen''}.
\item
  20110811 - Wired.it,
  \href{http://daily.wired.it/news/internet/2011/08/11/telecomix-hacker-egitto-tunisia-iran-13861.html}{``Telecomix,
  gli hacker di una volta''}.
\item
  20110607 - Computer Sweden,
  \href{http://computersweden.idg.se/2.2683/1.388901/han-praktiserar-yttrandefrihet}{``Han
  praktiserar yttrandefrihet''}.
\item
  2011060 - Dagens Nyheter,
  \href{http://www.dn.se/nyheter/varlden/sa-begransar-syrien-friheten-pa-internet}{"Så
  begränsar Syrien internet.}
\item
  20110407 - SR P1 Obs!, \href{http://t.sr.se/1zjt3PQ}{``Internet, Gud
  och människan''}.
\item
  20110407 - SR P1, \href{http://t.sr.se/PbptWr}{"Internet, ett nytt
  politiskt subjekt?}
\item
  20110407 - Hackerspaces Signal Radio,
  \href{http://signal.hackerspaces.org/archive/2011-04-07-2200-hacktivism-hour.mp3}{``Hacktivism
  hour''}.
\item
  20110322 - Jusektidningen,
  \href{http://www.tidningenkarriar.se/Arkivet/2011/3/Anarkistisk-hjalte/}{``En
  anarkistisk hjälte''}.
\item
  20110308 - Dagens Nyheter,
  \href{http://www.dn.se/vart-internet/vart-internet-hem/sa-kan-stater-hota-friheten-pa-natet/}{``Så
  kan stater hota friheten på nätet''}.
\item
  20110224 - Dagens Nyheter,
  \href{http://www.dn.se/vart-internet/vart-internet-hem/din-chatt-en-sakerhetsrisk/}{``Din
  chatt 1. en säkerhetsrisk''}.
\item
  20110221 - Sveriges Radio Studio ett,
  \href{http://t.sr.se/1cUlHfI}{"Studio Ett1. extrasändning om Libyen.}
\item
  20110214 - Sveriges Radio Studio ett,
  \href{http://t.sr.se/1D1zCXq}{"Iran 1. oppositionen uppmanar till
  demonstrationer.}
\item
  20110202 - PC-facile,
  \href{http://www.pc-facile.com/news/come_comunica_egitto_senza_internet/68959.htm}{``Come
  comunica l'Egitto senza Internet''}.
\item
  20110201 - New Scientist,
  \href{http://www.newscientist.com/blogs/onepercent/2011/02/egypt-remains-officially-offli.html}{``How
  Egypt is getting online without the internet''}.
\item
  20110131 - Deutsche Welle,
  \href{http://www.dw.com/en/european-activists-offer-dial-up-internet-to-get-egypt-back-online/a-14807049}{``European
  activists offer dial-up Internet to get Egypt back online''}.
\item
  20110131 - Radio Netherlands,
  \href{http://www.rnw.nl/english/article/dust-your-dialup-modem-contact-egypt}{``Dust
  off your old modem for Egypt''}.
\item
  20110131 - Huffington Post,
  \href{http://www.huffingtonpost.com/2011/01/29/anonymous-internet-egypt_n_815889.html}{``Anonymous
  Internet Users Team Up To Provide Communication Tools For Egyptian
  People''}.
\item
  20110129 - Sveriges Television,
  \href{http://www.youtube.com/watch?v=HRZ0QKyiFsE\&feature=related}{``Rapport''.}
\item
  20110114 - Hackerspaces Signal Radio,
  \href{http://signal.hackerspaces.org/archive/2011-01-13-2200-hacktivism-hour.mp3}{``Hackerspaces
  Signal Radio''}.
\item
  20110117 - Svenska Dagbladet,
  \href{http://www.svd.se/stjarnlackan-far-konkurrens}{``Stjärnläckan
  får konkurrens''}.
\item
  20101220 - Hufvudstadsbladet, ``En rörelse som inte går att stoppa''.
\item
  20101206 - SR P1 Studio ett,
  \href{http://t.sr.se/1CQkYCj}{``Nätfriheten hotas''}.
\item
  20101115 - SR P3 Brunchrapporten,
  \href{http://t.sr.se/RI5PTm}{``Cyberkrig, digital atombomb och
  hackersoldater''}.
\item
  20101017 - Sydsvenskan,
  \href{http://www.sydsvenskan.se/2010-10-17/trollen-harskar-i-natets-utmarker}{``Trollen
  härskar i nätets utmarker''}.
\item
  20100611 - NRLI TV @ Hacknight,
  \href{http://www.youtube.com/watch?v=VhmH6cJZS4k\&feature=related}{``How
  to Hack Politics. Part II Interview with Christopher Kullenberg''}.
\item
  20100601 - Sveriges Radio P3 Brunchrapporten,
  \href{http://t.sr.se/1cnEijY}{``Kommunikationsstopp för Ship to
  Gaza''.}
\item
  20090729 - Dagens Nyheter,
  \href{http://www.dn.se/kultur-noje/nyheter/internatet-som-vill-bygga-om-eu/}{``Internätet
  som vill bygga om EU''}.
\item
  20090427 - Peppar.fi, ``EU Röstar om internets framtid''.
\item
  20090422 - Dagens Nyheter,
  \href{http://www.dn.se/kultur-noje/musik/for-tidigt-ropa-hej-internet-inte-raddat/}{``För
  tidigt ropa hej 1. internet inte räddat''}.
\item
  20090416 - Dagens Nyheter
  \href{http://www.dn.se/kultur-noje/en-krigsforklaring-mot-internet/}{``En
  krigsförklaring mot internet''}.
\item
  20081215 - GU-Journalen,
  \href{http://www.gu-journalen.gu.se/english/News/News_detail/?contentId=855527}{``Journal
  across boundaries''}.
\item
  20081219 - Sheffield Live! Radio, \href{http://www.dcs.shef.ac.uk/}{}
\item
  20080626 - Proletären, \href{http://www.proletaren.se/inrikes/}{}
\end{enumerate}

\hypertarget{exhibitions-and-public-outreach-workshops}{%
\subsubsection{Exhibitions and public outreach
workshops}\label{exhibitions-and-public-outreach-workshops}}

\begin{enumerate}
\def\labelenumi{\arabic{enumi}.}
\tightlist
\item
  20180421 - Co-organiser and workshop instructor on
  \href{http://vetenskapsfestivalen.se/wp-content/uploads/2018/04/Vetenskapsfestivalen-\%C3\%96ppna-programmet-2018.pdf}{Citizen
  Science med miljösensorer för luftkvalitet}, Vetenskapsfestivalen
  (Gothenburg Science Fair), Gothenburg.
\item
  20180308 - Co-creator of
  \href{https://www.tekniskamuseet.se/pa-gang/invigning-digital-now-4-en-kod-som-forandrar-varlden-metoo/}{Digital
  Now \#4: En kod som förändrar världen \#metoo}, Tekniska muséet
  (National Museum of Science and Technology), Stockholm.
\end{enumerate}

\hypertarget{invited-speaker}{%
\subsubsection{Invited speaker}\label{invited-speaker}}

\begin{enumerate}
\def\labelenumi{\arabic{enumi}.}
\tightlist
\item
  20191205 - ``Who owns your data? A Conference about Big Data, privacy
  online and the future of internet'', Science Park Utrecht, Utrecht,
  The Netherlands, Organised by the Embassy of Sweden, The Hague.
\item
  20181129 -
  ``\href{https://www.lansstyrelsen.se/orebro/kalenderhandelser---orebro/2018-10-16-samverkansmote-om-artdata.html}{Samhällets
  användning av artdata}'', Samverkansmöte om Artdata, Länsstyrelsen i
  Örebro, Norrköping, Sweden.
\item
  20180926 - ``\href{https://www.swedenabroad.se/free-speech}{How does
  Technology Transform Media and Public Opinion? - about Digital
  Platforms and the Future of Free Speech}'', Science Park Utrecht,
  Utrecht, The Netherlands, Organised by the Embassy of Sweden, The
  Hague.
\item
  20180602 - ``The politics of Data in the intersection between Hacker
  Culture and Citizen Science'', Science \& Dissent Workshop, June 1-2,
  2018, University of Geneva, Switzerland
\item
  20171211 -
  \href{http://www.u-care.uu.se/kalendarium/evenemang/?eventId=29958}{``Öppen
  vetenskap och demokrati''}, Mötesplats U-CARE 1. Open Science, Uppsala
  University.
\item
  20171204 - ``Medborgarvetenskap och samverkan'', SSU Haga-Anneldal,
  Gotenburg.
\item
  20171108 -
  \href{http://gps400.gu.se/digitalAssets/1662/1662945_gps400-conference_final-programme.pdf}{"Collecting
  Analog Bulletin Board Messages Using a Citizen Science Methodology,
  GPS 400 Gothenburg Cultures on the town 1621-2021}.
\item
  20170915 -
  \href{http://patriksv.net/2017/08/program-and-inbjudna-deltagare-15-september-pa-kth/}{"Perspektiv
  på digitalt driven forskning inom humaniora'}, Royal Institute of
  Technology (KTH), Stockholm.
\item
  20170426 -
  \href{https://ecsa.citizen-science.net/sites/default/files/draft_agenda_second_workshop_appsplatforms.pdf}{``Collecting
  Social Science Data with Smartphone Apps and School Children''}, ECSA
  workshop ``Defining Principles for Mobile Apps and Platforms
  Development in Citizen Science'', Gothenburg, Sweden.
\item
  20170330 - ``Vad händer om vi inte kan enas om vad som är sant?'',
  \href{http://flov.gu.se/aktuellt/Nyheter/fulltext//oppet-hus-pa-flov-med-panelsamtal-om-alternativa-fakta-och-informationsbubblor.cid1428426}{FLoV,
  Göteborgs Universitet}.
\item
  20170329 - ``Vad är på gång inom citizen science?'',
  \href{http://www.vinnova.se/sv/Aktuellt--publicerat/Kalendarium/2017/170329-Informationsmote-Vetenskap-med-och-for-samhallet/}{Informationsmöte
  Vetenskap med och för samhället Horisont 2020}, Vinnova, Stockholm.
  Lecture available on \href{https://youtu.be/HHW0j7Zo5E0}{Youtube}.
\item
  20161214 - ``Digital methods 1. Theory, Practice and Ethics'', Higher
  seminar, Karlstad University.
\item
  20161116 -
  \href{http://www.stadsbiblioteket.nu/tusen-plataer/}{``Tusen
  platåer''}, Göteborg, Stadsbiblioteket.
\item
  20161028 - ``Statens offentliga utredningar digitaliserade'', Högre
  seminarium, Idéhistoria, Lunds universitet.
\item
  20161012 - \href{https://v-a.se/events/va-dagen-2016/}{``VA-Dagen 2016
  om öppen vetenskap''}, Stockholm. "Video:
  \url{https://youtu.be/CUNnddKvZ9o?t=43m52s}.
\item
  20160922 -
  \href{http://flov.gu.se/aktuellt/Nyheter/fulltext//sju-filosofer-forelaser-pa-bokmassan-.cid1403852}{``Kan
  alla forska?''}, Forskartorget, Bokmässan, Göteborg.
\item
  20160921 -
  \href{http://kompetensutveckling.adm.gu.se/seminar/detail/2260}{``Att
  nå ut via sociala medier''}, University of Gothenburg.
\item
  20160912 -
  \href{http://www.kb.se/aktuellt/utbildningar/2016/Kulturarvet-som-ettor-och-nollor--Del-3-Digital-humaniora/}{``Kulturavet
  som ettor och nollor''}, National Library of Sweden.
\item
  20160209 -
  \href{http://www.abfgoteborg.org/index.php/archive/2016/170-filosofiscenen-2016/2208-sarah-ahmed-lycka-och-falskt-medvetande}{``Sara
  Ahmed 1. lycka, olycka och falskt medvetande''}, ABF, Göteborg,
  Sweden.
\item
  20150929 -
  \href{http://hum.gu.se/aktuellt/Nyheter/fulltext//nackrostimmen--spionen-i-fickan-som-overvakar-oss.cid1324880}{``Spionen
  i Fickan''}, Näckrostimmen, University of Gothenburg.
\item
  20151209 - \href{http://fhp.nu/tusenplataer}{``Samtal om Tusen
  Platåer''}, Bio Rio, Stockholm, Sweden.
\item
  20151020 -
  \href{http://www.kb.se/aktuellt/evenemang/2015/SOUhack/}{``Statens
  röst digitaliserad''}, National Library of Sweden.
\item
  20141126 -
  \href{http://www.wherevent.com/detail/Goteborgs-Konsthall-TUNNELPOLITIK-Forelasning-med-Christopher-Kullenberg}{Tunnelpolitik},
  Konsthallen, Gothenburg.
\item
  20141122 -
  \href{https://www.facebook.com/events/622473101197803/}{``Re-building
  a common internet''}, Verkko Suljettu, Helsinki, Finland.
\item
  20140604 -
  \href{http://www.urbsec.se/digitalAssets/1483/1483366_program-urbsec-konferens-2014-06-04.pdf}{``Forskning,
  allmänhet, innovation -- Crowd Science som nytt forskningsparadigm''},
  URBSEC, Gothenburg, Sweden.
\item
  20130223 -
  \href{http://www.frilansjournalisten.nu/2013/01/sasong-for-arsmoten/}{``Kommunikation
  och nätet''}, Smålands Frilansklubb, Växjö, Sweden.
\item
  20121207 -
  \href{http://letstudio.gu.se/svenska/aktuellt/nyheter/n/christopher-kullenberg-inbjuden-som-talare-pa-internationell-konferens.cid1111971}{``Does
  Internet freedom have a price? Examples from the Arab Spring''},
  \href{http://www.hre2012.uj.edu.pl/invited-speakers}{The Third
  International Conference on Human Rights Education, Jagiellonian
  University, Krakow, Poland}.
\item
  20121105 - \href{https://www.youtube.com/watch?v=Zo24Qy_PU8I}{``Vilken
  roll spelar de nya internetbaserade medierna i samhällsförändrande
  aktiviteter?''}, Sigmas inspirationsdag, Gothenburg, Sweden.
\item
  20120822 -
  \href{https://transmediale.de/content/resource-002-out-place-out-time}{``reSource
  002: Out of Place, Out of Time''}, Transmediale, Berlin, Germany.
\item
  20120509 -
  \href{http://pol.gu.se/aktuellt/kalendarium/aktuellt_detalj/?eventId=1777397828}{``The
  Arabic Spring. Internet and social media 1. crucial tools in
  organizing the civil uprisings in Tunis and Egypt.''}, Brännpunkt
  Europa, University of Gothenburg.
\item
  20120301 -
  \href{http://www.youtube.com/watch?v=-HYfVmanye8}{``Humanistisk
  forskning och engagemang''}, Akademisk kvart, University of
  Gothenburg, Sweden.
\item
  20111124 - \href{https://www.youtube.com/watch?v=zWROWpMaKmE}{``Our
  Internet 1. Our Rights, Our Freedoms''}, Council of Europe, Vienna,
  Austria.
\item
  20111119 - ``What will be the needs of tomorrow's leaders'', Swedish
  Embassy Cairo, Cairo, Egypt.
\item
  20111022 -
  \href{http://www.youtube.com/watch?v=6Mi0g93ModU}{``Nätaktivism -- nya
  möjligheter men också faror''}. Svenska FN-förbundet, Göteborg,
  Sweden.
\item
  20111304 -
  \href{http://www.rsf-ch.ch/anonymat-des-communications-s\%C3\%A9curit\%C3\%A9-des-donn\%C3\%A9es-protection-des-sources-0}{``Anonymat
  des communications, sécurité des données, protection des sources''},
  Reporters Sans Frontieres, Geneva, Switzerland.
\item
  20100504 - ``Greens/EFA and Activists Workshop on ACTA and the Public
  interest'', European Parliament, Brussels, Belgium.
\item
  20090604 -
  \href{https://issuu.com/deutsche-welle/docs/program-deutsche-welle-global-media-forum-2009}{``Information
  Technology: Provoking or Preventing Conflict''}, Deutsche Welle Global
  Media Forum, Bonn, Germany.
\item
  20091107 - \href{https://vimeo.com/10286077}{``Net Neutrality,
  Surveillance and how to Re-build Politics''}, Free Society Conference,
  Gothenburg, Sweden.
\item
  20091103 -
  \href{https://internetdagarna.se/arkiv/2009/program-2009/3-november.html}{``Revision
  av telekompaketet''}, Internetdagarna, Stockholm, Sweden.
\item
  20091028 - ``Invited speaker by the Ministry of Enterprise, Energy and
  Communications on the implementation of the Telecoms Package'',
  Government of Sweden, Stockholm, Sweden.
\item
  20090111 -
  \href{https://www.youtube.com/watch?v=G9cXIKvywvs}{``Lightning talk:
  Resistance Studies Magazine''}, 25th Chaos Communication Congress,
  Berlin, Germany.
\item
  20081108 -
  \href{http://www.fiff.de/veranstaltungen/fiff-jahrestagungen/fiff-jahrestagung-2008-krieg-und-frieden-digital/Programmheft.pdf/at_download/file}{``The
  social impact of IT''}, Krieg und Frieden Digital, Aachen, Germany.
\end{enumerate}

\hypertarget{formal-education-teaching-in-higher-education-huxf6gskolepedagogik}{%
\subsection{Formal education, Teaching in higher education
(Högskolepedagogik)}\label{formal-education-teaching-in-higher-education-huxf6gskolepedagogik}}

\begin{enumerate}
\def\labelenumi{\arabic{enumi}.}
\tightlist
\item
  \href{http://files.christopherkullenberg.se/hogskolepedagogikonline.pdf}{2006.
  Teaching and Learning in Higher Education}
  (\href{http://files.christopherkullenberg.se/PE0940.pdf}{Högskolepedagogik
  för lärare, PE0940}), 7.5 credits.
\item
  \href{http://files.christopherkullenberg.se/hogskolepedagogikonline.pdf}{2006.
  Supervision in Postgraduate Education}
  (\href{http://files.christopherkullenberg.se/PD0180.pdf}{Högskolepedagogisk
  handledning, PD0180}), 7.5 credits.
\item
  \href{http://kursplaner.gu.se/svenska/HPE102.pdf}{2016. Formal credits
  for the course HPE102 Teaching and learning in Higher Education 2:
  Discipline Specific Pedagogic}.
\item
  \href{http://kursplaner.gu.se/svenska/HPE103.pdf}{2018. HPE103:
  Teaching and Learning in Higher Education 3: Applied Analysis, 5
  Credits, Second Cycle}
\end{enumerate}

\hypertarget{courses-thesis-supervisionexamination-graduate-level}{%
\subsection{Courses \& Thesis supervision/examination, graduate
level}\label{courses-thesis-supervisionexamination-graduate-level}}

\hypertarget{theory-of-science-courses}{%
\subsubsection{Theory of Science:
Courses}\label{theory-of-science-courses}}

\hypertarget{course-administrator-and-teacher}{%
\paragraph{Course administrator and
teacher}\label{course-administrator-and-teacher}}

\begin{enumerate}
\def\labelenumi{\arabic{enumi}.}
\tightlist
\item
  2019 VT1500 ``Medborgarforskning: teori och praktik'', University of
  Gothenburg.
\item
  2017 (Course administrator together with Mats Dahlström) Digital
  Research Methods, distance course 15 credits BMBD116h, University of
  Borås.
\item
  2013
  \href{http://files.christopherkullenberg.se/kursplaner/FHV241Strimman.pdf}{``FHV241
  The Strand: People, Knowledge and Public Health Practice''}, 6
  credits, (Strimman, moment 1: Tre konstruktioner av människan och
  folkhälsoarbete i praktiken), 1.5 credits, Second cycle, University of
  Gothenburg.
\item
  2007 - 2014. (Course administrator 2012 - 2013).
  \href{http://files.christopherkullenberg.se/kursplaner/NTH001_Teoretiska_och_historiska_perspektiv_pa\%cc\%8a_naturvetenskap_10512.pdf}{"NTH001
  Theoretical and Historical Perspectives on Science, 7.5 credits},
  First cycle, University of Gothenburg.
\item
  2014
  \href{http://files.christopherkullenberg.se/kursplaner/VT1101_Introduktion_i_klassisk_vetenskapsteori__grundkurs_13185.pdf}{``VT1101
  Classical Theory of Science, Introductory Course (Introduktion till
  klassisk vetenskapsteori)''}, 15 credits, First cycle, University of
  Gothenburg.
\item
  2014
  \href{http://files.christopherkullenberg.se/kursplaner/VT2106_Klassisk_vetenskapsteori_13187.pdf}{``VT2106
  Classical Theory of Science (Klassisk vetenskapsteori)''}, 15 credits,
  Second cycle level, University of Gothenburg.
\end{enumerate}

\hypertarget{media-communication-studies-strategic-communication-courses}{%
\subsubsection{Media \& Communication Studies, Strategic Communication:
Courses}\label{media-communication-studies-strategic-communication-courses}}

\hypertarget{course-administrator-and-teacher-1}{%
\paragraph{Course administrator and
teacher}\label{course-administrator-and-teacher-1}}

\begin{enumerate}
\def\labelenumi{\arabic{enumi}.}
\tightlist
\item
  2019 (Course administrator) ``KT2109 - Strategisk kommunikation'', 7,5
  credits, Second cycle, University of Gothenburg.
\item
  2019 (Course administrator) ``KT1102 - Kommunikation i nya och sociala
  medier - Introduktion'', 7,5 credits, First cycle, University of
  Gothenburg.
\item
  2018 (Course administrator) ``KT2104 H18 Organisationskommunikation 1
  / Organizational Communication'', 7,5 credits, Second cycle,
  University of Gothenburg.
\item
  2018, 2019 (Course administrator) ``KT2501 V18 Magisteruppsats'', 15
  credits, Second cycle, University of Gothenburg.
\item
  2012, 2013, 2017, 2018, 2019 (Course administrator)
  \href{http://files.christopherkullenberg.se/kursplaner/KT2102.pdf}{``Communication
  in New and Social Media''}, 7.5 credits, Second cycle, University of
  Gothenburg.
\item
  2005 (Course administrator) ``Media, Individual and Every-day Life'',
  7.5 credits, Basic level, University West.
\item
  2005 (Course administrator) ``Strategic Communication'', 7.5 credits,
  Basic level, University West.
\end{enumerate}

\hypertarget{masters-programme-for-communication-officers-in-the-public-sector-thesis-supervision-master-magister-level}{%
\subsubsection{Master's Programme for communication officers in the
public sector: Thesis supervision, Master (Magister)
level\}}\label{masters-programme-for-communication-officers-in-the-public-sector-thesis-supervision-master-magister-level}}

\begin{enumerate}
\def\labelenumi{\arabic{enumi}.}
\tightlist
\item
  Gabriela Velasquez Ulloa (2018)
  \href{https://gupea.ub.gu.se/handle/2077/58024}{``Hur når vi ut? Om
  att skapa engagemang på sociala medier''}, University of Gothenburg.
\item
  Emelie Engström (2018)
  \href{https://gupea.ub.gu.se/handle/2077/58030}{``Goa patienter och
  kompetent personal? En innehålls- och textanalys av Sahlgrenska
  Universitetssjukhusets värdeord på Instagram''}, University of
  Gothenburg.
\item
  Clara Tortosa (2017)
  \href{https://gupea.ub.gu.se/handle/2077/54970}{``Vem är göteborgaren
  i Vårt Göteborg? - En diskursanalys av Vi och dem i en
  kommuntidning''}, University of Gothenburg.
\item
  Peter Häggbom Norrby (2017)
  \href{https://gupea.ub.gu.se/handle/2077/54948}{``Medborgaren i
  kommunens kommunikationspolicy - En jämförelse mellan tio kommuner''},
  University of Gothenburg.
\item
  Louise Sjöström (2017), ``Från kulturpolitik till kommunikation - En
  analys av hur fem verksamheters statliga och kommunala uppdrag omsätts
  i kommunikationen'', University of Gothenburg.
\item
  Joacim Schmidt (2017)
  \href{https://gupea.ub.gu.se/bitstream/2077/52545/1/gupea_2077_52545_1.pdf}{``Facebooks
  möjligheter och risker: En studie om svenska kommuners användande av
  Facebook''}, University of Gothenburg.
\item
  Sofie Andersson (2013) ``Det kommunikativa målarbetet -- en kommunal
  budgetresa från politiker till medarbetare'', University of
  Gothenburg.
\item
  Malin Arosilta (2013) ``Demokratiska värden som varumärke - En studie
  om''Göteborg 2021``.'', University of Gothenburg.
\item
  Anna Cöster (2013) ``Att vara kyrka på internet. Svenska kyrkan i
  sociala medier.'', University of Gothenburg.
\item
  Charlotta Dahlberg (2013) ``Utbildning till salu - gymnasieskolors
  kommunikativa strategier.'', University of Gothenburg.
\item
  Jenny Fogelberg (2013) ``Kommunen i sociala medier - transparens och
  medborgardialog i Varbergs kommun.'', University of Gothenburg.
\item
  Daniel Alexander Jansson (2013) ``Felaktiga Förväntningar - en studie
  av Arbetsförmedlingens kommunikation mot ungdomar'', University of
  Gothenburg.
\item
  Sven Lindström (2013) ``Vad håller vi på med? - En kartläggning av
  kommunikationsprocessen kring kampanjen''Vad håller du på med?''",
  University of Gothenburg.
\item
  Sunanda Malm (2013) " Målförståelse, kommunikation och
  varumärkesarbete på en svensk myndighet, sett ur ett
  medarbetarperspektiv"", University of Gothenburg.
\item
  Maria Pettersson (2013) "``Vissa saker är bara jävligt svårt att prata
  om'' - Om ungas användning av internet för terapeutiska samtal.",
  University of Gothenburg.
\item
  Annie Sjölund (2013) ``Skall man skjuta får man ju vårda också'' - En
  studie om sjuksköterskors uppfattningar om Försvarsmakten", University
  of Gothenburg.
\item
  Agneta Slonawski (2013) "''Jag tror man mest söker information hos
  kompisar eller på nätet.'' - Förutsättningar för att nå unga med
  information om alkohol och droger för Mini-Maria Göteborg.",
  University of Gothenburg.
\item
  Karin Svenner (2013) ``Synen på kommunikation - en kvalitativ
  innehållsanalys av femton kommunikationspolicyer i offentlig
  förvaltning'', University of Gothenburg.
\item
  Lovisa Vasiliou (2013) ``1.000 bilder av Värmland - En studie av
  Region Värmlands kommunikationsprocess inför skrivandet av
  Värmlandsstrategin'', University of Gothenburg.
\item
  Jörgen Wade (2013) ``Förväntningar på socialt intranät och vilken syn
  på organisationskommunikation dessa förväntningar reflekterar'',
  University of Gothenburg.
\end{enumerate}

\hypertarget{media-and-communication-studies-thesis-supervision-bachelor-level}{%
\subsubsection{Media and communication studies: Thesis supervision,
bachelor
level}\label{media-and-communication-studies-thesis-supervision-bachelor-level}}

\begin{enumerate}
\def\labelenumi{\arabic{enumi}.}
\tightlist
\item
  Charlotta Windeman och Linda Andreasson (2006)
  \href{http://urn.kb.se/resolve?urn=urn:nbn:se:hv:diva-846}{``Meningen
  måste ju vara att få folk till butiken: en kvalitativ studie av ICA
  respektive Coop:s profiler och images''}, University West, Media \&
  Communication studies.
\item
  Andreas Johansson och Ellinor Wetterblad (2006)
  \href{http://urn.kb.se/resolve?urn=urn:nbn:se:hv:diva-846}{``Organisationers
  medierelationer : en kvantitativ kartläggning av organisationers
  mediekontakt och medieverktyg''}, University West, Media \&
  Communication studies.
\item
  Lars Karlsson (2006)
  \href{http://www.uppsatser.se/uppsats/7e12f0e57f/}{``Primitiv och utan
  identitet : en kvalitativ analys av synen på''den andre" i Metros
  pratbubbletävling"}, University West, Media \& Communication studies.
\item
  Tobias Hagrenius (2006)
  \href{http://urn.kb.se/resolve?urn=urn:nbn:se:hv:diva-851}{``Samma
  nyheter i olika tidningar? en kvalitativ innehållsanalys på tre
  gratistidningar''}, University West, Media \& Communication studies.
\item
  Jelena Siric, Lorans Zaya (2005) ``Släpp in mig, yao! En studie om den
  kontroversiella tidningen Gringo'', University West, Media \&
  Communication studies.
\item
  Jeanette Borg, Karolina Colliander (2005) ``Kolla - Kollektivtrafik åt
  alla'', University West, Media \& Communication studies.
\item
  Annika Jonasson och Tina Nyrén (2005) ``Terrorism är kommunikation. En
  diskursanalys av Aftonbladet, Dagens Nyheter och Svenska Dagbladets
  rapportering av terrordåden vid München 1972, Lockerbie 1988 och World
  Trade Center 2001'', University West, Media \& Communication studies.
\item
  Daniel Sjöberg (2004) ``Ideologi i amerikanska programprodukter - En
  receptionsanalys av programmet L-word baserad på kvalitativa
  intervjuer med elva respondenter''. University West, Media and
  Communication studies.
\item
  Christina Axelsson och Helena Johansson (2004) ``LunarStorm -
  Framtidens fritidsgård? En kvalitativ studie om unga människors
  användning av Internetcommunities och deras inställningar till
  betaltjänster'', University West, Media and Communication studies.
\end{enumerate}

\hypertarget{masters-programme-for-communication-officers-in-the-public-sector-thesis-examination-masters-degree-magister.}{%
\subsubsection{Master's Programme for communication officers in the
public sector: Thesis examination, master's Degree
(Magister).}\label{masters-programme-for-communication-officers-in-the-public-sector-thesis-examination-masters-degree-magister.}}

\begin{enumerate}
\def\labelenumi{\arabic{enumi}.}
\tightlist
\item
  Emilia Vermelin (2019) ``Dömd på förhand: (o)dialogisk kommunikation
  och epistemisk orättvisa.'', University of Gothenburg.
\item
  Malin Fallgren (2019) ``Kan kultur kommunicera mellan myndighet och
  medborgare? - En kommunikationsstrategisk fallstudie av Lights in
  Biskopsgården'', University of Gothenburg.
\item
  Lina Ahlstedt (2019) ``Lyssnar dom??'', University of Gothenburg.
\item
  Katarina von Sydow (2018)
  \href{https://gupea.ub.gu.se/handle/2077/58013}{"Angereds närsjukhus
  kommunikation av folkhälsa -- en fallstudie av `Rökfria lekplatser'
  och `Angereds hjältar'}, University of Gothenburg.
\item
  Sofie Alnäs (2018) \href{https://gupea.ub.gu.se/handle/2077/58021}{``I
  skärningspunkten mellan lokal och central styrning. En undersökning av
  kommunikatörens roll på sex institutioner vid Göteborgs
  universitet''}, University of Gothenburg.
\end{enumerate}

\hypertarget{information-architecture-studies-thesis-examination-bachelors-degree}{%
\subsubsection{Information architecture studies: Thesis examination,
bachelor's
Degree}\label{information-architecture-studies-thesis-examination-bachelors-degree}}

\begin{enumerate}
\def\labelenumi{\arabic{enumi}.}
\tightlist
\item
  Jonna Axelsson (2017) ``Ideella organisationers webbplatser blir
  användbara - Hur användarupplevelsen kan förbättras på
  leadersodrabohuslan.se med fokus på användbarhet'', Bachelor's Thesis,
  Information Architecture, University of Borås.
\item
  Christine Thelin (2017) ``Skapa bättre användarupplevelse och öka
  besöken - Utvärdering och utvecklingsförslag åt en webbplats inom
  cafébranschen'', Bachelor's Thesis, Information Architecture,
  University of Borås.
\end{enumerate}

\hypertarget{strategic-information-and-communication-thesis-supervision-masters-degree}{%
\subsubsection{Strategic information and communication: Thesis
supervision, master's
Degree}\label{strategic-information-and-communication-thesis-supervision-masters-degree}}

\begin{enumerate}
\def\labelenumi{\arabic{enumi}.}
\tightlist
\item
  Emma Håkansson (2018) ``Att kommunicera strategiskt - En kvalitativ
  studie av strategisk kommunikation i tolv svenska kommuner'',
  University of Borås.
\item
  Emma Lidell (2018) ``The United Nations and the Sustainable
  Development Goals - The importance of communicating sustainability'',
  University of Borås.
\item
  Linn Lundström (2018) ``Att stimulera hållbara attityder och beteenden
  i dagens konsumtionssamhälle'', University of Borås.
\end{enumerate}

\hypertarget{strategic-information-and-communication-thesis-examination-masters-degree}{%
\subsubsection{Strategic information and communication: Thesis
examination, master's
Degree}\label{strategic-information-and-communication-thesis-examination-masters-degree}}

\begin{enumerate}
\def\labelenumi{\arabic{enumi}.}
\tightlist
\item
  Viktoria Lagerkvist (2018) ``Samspel eller kamp? En kritisk analys av
  bibliotekschefers strategiska kommunikation i den kommunala politiska
  verkligheten'', Magisteruppsats, University of Borås.
\item
  Julia Ahlqvist (2017) ``Sambandet mellan kommunikativt ledarskap och
  arbetsmotivation hos medarbetare i servicebranschen'',
  Magisteruppsats, University of Borås.
\end{enumerate}

\hypertarget{evidence-basing-practice-theory-context-masters-level-magister-thesis-supervision}{%
\subsubsection{Evidence-basing: practice, theory, context: Master's
level (Magister): Thesis
supervision}\label{evidence-basing-practice-theory-context-masters-level-magister-thesis-supervision}}

\begin{enumerate}
\def\labelenumi{\arabic{enumi}.}
\tightlist
\item
  Emelie Askaner (2018) "Implementering och standardisering:
  Undersökning av en standardiserings påverkan på arbetsterapeuter
  professionella status - Förväntningar och farhågor inför en
  implementeringsplan av AusTOMs, Magisteruppsats, University of
  Gothenburg.
\end{enumerate}

\hypertarget{evidence-basing-practice-theory-context-masters-level-magister-thesis-examination}{%
\subsubsection{Evidence-basing: practice, theory, context: Master's
level (Magister): Thesis
examination}\label{evidence-basing-practice-theory-context-masters-level-magister-thesis-examination}}

\begin{enumerate}
\def\labelenumi{\arabic{enumi}.}
\tightlist
\item
  Annika Linell (2015) ``Att bryta ny mark. En analys av
  Skolforskningsinstitutets första steg inom utbildningsvetenskaplig
  kumulativitet.'', Magisteruppsats, University of Gothenburg.
\end{enumerate}

\hypertarget{courses-thesis-supervision-postgraduate-level}{%
\subsection{Courses \& Thesis supervision, postgraduate
level}\label{courses-thesis-supervision-postgraduate-level}}

\hypertarget{ph.d-thesis-supervision}{%
\subsubsection{Ph.D-thesis supervision}\label{ph.d-thesis-supervision}}

\begin{enumerate}
\def\labelenumi{\arabic{enumi}.}
\tightlist
\item
  2017 - ongoing.
  \href{http://files.christopherkullenberg.se/Beslut\%20byte\%20handledare\%20J\%20Vikstrom.pdf}{Co-supervisor
  for Licentiate thesis of Jörgen Vikström}, Theory of Science,
  University of Gothenburg.
\item
  2016 - 2017. Co-supervisor for Ph.D.-candidate Erik Joelsson, Theory
  of Science, University of Gothenburg.
  \href{https://gupea.ub.gu.se/handle/2077/51493}{Thesis completed in
  February 2017}.
\end{enumerate}

\hypertarget{teaching-post-graduate-level}{%
\subsubsection{Teaching, post-graduate
level}\label{teaching-post-graduate-level}}

\begin{enumerate}
\def\labelenumi{\arabic{enumi}.}
\tightlist
\item
  (2018-03-14), Guest lecture, ``Digital methods'', \emph{Digital and
  Visual Methods}, Chalmers University of Technology, Dept. for Science,
  Technology and Society.
\item
  (2017), Guest lecture, ``Outreach and cooperation with society at
  large'',
  \href{http://hum.gu.se/utbildning/forskarniva/introduktionsvecka-for-doktorander-2017}{Introduktionsvecka
  för doktorander}, University of Gothenburg.
\item
  (2016), Guest lecture ``Citizen Science and Biodiversity Research'',
  BioEnv higher education, Department of Biological and Environmental
  Sciences, University of Gothenburg.
\item
  (2012), Guest lecture, ``Net politics, process philosophy and
  Deleuze'', Theories of Practical Knowing, HDK/Högskolan för Design och
  Konsthantverk, Göteborg.
\item
  (2012), Guest lecture and seminar, ``Tyst kunskap'', 2012 års
  Marstrandsseminarium, Teori \& metod Tystnad, tradition eller
  ``anything goes''?, Varberg.
\end{enumerate}

\hypertarget{teaching-materials}{%
\subsection{Teaching materials}\label{teaching-materials}}

\hypertarget{electronic-material}{%
\subsubsection{Electronic material}\label{electronic-material}}

\begin{enumerate}
\def\labelenumi{\arabic{enumi}.}
\tightlist
\item
  \href{http://digitalametoder.science}{digitalametoder.science - Tools
  and tutorials for digital methods in the humanities and social
  sciences}.
\end{enumerate}

\hypertarget{other-pedagogical-experiences}{%
\subsection{Other pedagogical
experiences}\label{other-pedagogical-experiences}}

\begin{enumerate}
\def\labelenumi{\arabic{enumi}.}
\tightlist
\item
  2018 PI for
  \href{https://gup.ub.gu.se/publication/273029}{``Alumnundersökning
  Kommunikatörsprogrammet'', Alumni Study Report}.
\item
  2018 - ongoing. Programme Co-ordinator for the Master's Programme for
  communication officers in the public sector, 60 - 120 credits,
  University of Gothenburg.
\item
  \href{http://files.christopherkullenberg.se/studierektoronline.pdf}{Study
  Administrator (Studierektor), Spring semester 2014}.
\item
  \href{http://files.christopherkullenberg.se/IntygREGU.pdf}{2012 -
  2014. Member of Rådet för Europastudier vid Göteborgs universitet
  (REGU)}.
\end{enumerate}

\end{document}
